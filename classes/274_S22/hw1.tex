\documentclass[11pt]{article}
\usepackage[top=1in, bottom=1in, left=1in, right=1in]{geometry}

\usepackage{amsmath}
\usepackage{amssymb}

\usepackage{soul}

\usepackage{hyperref}

\title{Howard Math 274, HW\# 1, \\ {\normalsize Spring 2022; Instructor: Sam Hopkins; Due: \st{Friday, February 11th} Monday, February 14th}}
\date{}

\begin{document}

\maketitle

\thispagestyle{empty}

\vspace{-1.5cm}

\begin{enumerate}

\item A \emph{$k$-ary necklace of length $n$} is a rotation equivalence class of colorings of the vertices of an $n$-gon with $k$ colors. Use unweighted P\'{o}lya counting to show that the number of $k$-ary necklaces of length~$n$ is
\[ \frac{1}{n}\sum_{d\mid n} \varphi(d) k^{\frac{n}{d}}.\]
This formula uses some notation from number theory: $d \mid n$ means ``$d$ divides $n$''; and $\varphi(d)$ is \emph{Euler's totient function}, the number of $1 \leq j \leq d$ with $\gcd(d,j)=1$.

\item Continuing the previous problem, now using weighted P\'{o}lya counting: how many ways, up to rotation, can the vertices of a hexagon be colored with $2$ red, $2$ green, and $2$ blue vertices?

\item There are $24$ orientation-preserving symmetries of a cube-- they are all spatial rotations. Use unweighted P\'{o}lya counting to give a formula for the number of ways, up to orientation-preserving symmetries, to color the faces of a cube with $k$ colors. 

{\bf Hint 1}: Your formula should be a polynomial in $k$.

{\bf Hint 2}: This group of symmetries is \emph{abstractly} isomorphic to the symmetric group $S_4$ (but of course there are \emph{six}, not four, faces of a cube); for more information on this group see for instance the Wikipedia page \url{https://en.wikipedia.org/wiki/Octahedral_symmetry}.

\item Continuing the previous problem, now using weighted P\'{o}lya counting: how many ways, up to orientation-preserving symmetries, can the faces of a cube be colored with $2$ red, $2$ green, and $2$ blue faces?

\item Let $\mathcal{M}_{n \times m}(k)$ be the set of $n \times m$ matrices with entries from the set $\{1,2,\ldots,k\}$. For example,
\[ \begin{pmatrix} 2 & 3 & 4 & 2 \\
3 & 1 & 2 & 3 \\
4 & 3 & 5 & 2 \end{pmatrix} \in \mathcal{M}_{3 \times 4}(5).\]
The symmetric group $S_n$ acts on $\mathcal{M}_{n \times m}(k)$ by permuting rows: e.g., for $\sigma = (1,2)(3) \in S_3$,
\[ \sigma \cdot \begin{pmatrix} 2 & 3 & 4 & 2 \\
3 & 1 & 2 & 3 \\
4 & 3 & 5 & 2 \end{pmatrix} = \begin{pmatrix} 3 & 1 & 2 & 3 \\
2 & 3 & 4 & 2 \\
4 & 3 & 5 & 2 \end{pmatrix} .\]
Let $\widetilde{\mathcal{M}}_{n \times m}(k)$ denote the set of $S_n$-equivalence classes of $\mathcal{M}_{n \times m}(k)$.  Give a formula (in terms of $n$, $m$, and $k$) for $\#\widetilde{\mathcal{M}}_{n \times m}(k)$.

{\bf Hint}: To simplify your formula you may use the fact, which we proved last semester, that the (unsigned) Stirling numbers of the 1st kind $c(n,j) := \# \{\sigma \in S_n\colon \textrm{$\sigma$ has $j$ cycles}\}$ have generating function $\sum_{j=1}^{n} c(n,j)t^j = t(t+1)\cdots (t+n-1)$.

{\bf Hard bonus problem, just to think about}: What if I'm allowed to independently permute both \emph{rows and columns} of the matrix?

\end{enumerate}


\end{document}