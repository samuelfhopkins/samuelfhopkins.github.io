\documentclass[11pt]{article}
\usepackage[top=1in, bottom=1in, left=1in, right=1in]{geometry}

\usepackage{amsmath}
\usepackage{amssymb}

\title{Math 211 (Modern Algebra II), HW\# 4, \\ {\normalsize Spring 2025; Instructor: Sam Hopkins; Due: Wednesday, March 19th}}
\date{}

\begin{document}

\maketitle

\thispagestyle{empty}

In this homework, all roots of unity are meant over the complex numbers $\mathbb{C}$.

\begin{enumerate}

\item Let $1 \leq k \leq n$ be integers. Prove that $k$ is a unit in the ring $\mathbb{Z}/n\mathbb{Z}$ if and only if $\mathrm{gcd}(k,n)=1$. Conclude that the following quantities are all equal to \emph{Euler's totient function} $\varphi(n)$:
\begin{itemize}
\item the order of the group of units $(\mathbb{Z}/n\mathbb{Z})^{\times}$;
\item the number of generators of $(\mathbb{Z}/n\mathbb{Z},+)$;
\item the number of primitive $n$th roots of unity;
\item the degree of the $n$th cyclotomic polynomial $\Phi_x(n)$;
\item $[\mathbb{Q}(\zeta_n)\colon\mathbb{Q}]$, where $\zeta_n=e^{\frac{2\pi i}{n}}$ is a primitive $n$th root of unity.
\end{itemize}

\item Let $\Phi_n(x)$ denote the $n$th cyclotomic polynomial. Prove the following about these $\Phi_n(x)$:
\begin{enumerate}
\item If $n=p$ is prime, then $\Phi_p(x) = 1+x+x^2+\cdots+x^{p-1}$.
\item If $n=2p$ is twice an odd prime $p$, then $\Phi_{2p}(x) = \Phi_p(-x)$.
\item If $n=p^k$ is a power of the prime $p$, then $\Phi_{p^k}(x) = \Phi_{p}(x^{p^{k-1}})$.
\end{enumerate}

\item Let $n > 2$, and let $\zeta_n$ be a primitive $n$th root of unity. Prove that $[\mathbb{Q}(\zeta_n+\zeta_n^{-1}):\mathbb{Q}] = \varphi(n)/2$. {\bf Hint:} It suffices to show $[\mathbb{Q}(\zeta_n):\mathbb{Q}(\zeta_n+\zeta_n^{-1})]=2$ (why?). To show $[\mathbb{Q}(\zeta_n):\mathbb{Q}(\zeta_n+\zeta_n^{-1})] \leq 2$, find a degree two polynomial $f(x) \in \mathbb{Q}(\zeta_n+\zeta_n^{-1})[x]$ which has $\zeta_n$ as a root. To show that $\mathbb{Q}(\zeta_n+\zeta_n^{-1}) \neq \mathbb{Q}(\zeta_n)$, think about which of these are subfields of $\mathbb{R}$ versus $\mathbb{C}$.

\item \begin{enumerate}
\item Let $f(x) = ax^3+bx^2+cx+d \in \mathbb{Q}[x]$ be a cubic polynomial (so $a \neq 0$). Show that the polynomial $\frac{1}{a} \cdot f(x-\frac{b}{3a})$ has the form $x^3+px+q$ for $p,q\in \mathbb{Q}$.
\item Let $f(x) = x^3+px+q\in \mathbb{Q}[x]$. Show that one root of~$f(x)$ has the form $x=\sqrt[3]{A}+\sqrt[3]{B}$ where 
\[ A = \frac{-q}{2} + \sqrt{\frac{q^2}{4}+\frac{p^3}{27}}, \quad B = \frac{-q}{2} - \sqrt{\frac{q^2}{4}+\frac{p^3}{27}}.\]
(This solution to the cubic equation is often called \emph{Cardano's formula}.) \\
{\bf Hint:} First notice (and explain why!) that with $x=\sqrt[3]{A}+\sqrt[3]{B}$ we get 
\[ x^3+px+q = A+B+(3\sqrt[3]{AB}+p)(\sqrt[3]{A}+\sqrt[3]{B})+q.\]
Then what can you say about the term $(3\sqrt[3]{AB}+p)$?
\end{enumerate}

\end{enumerate}


\end{document}
