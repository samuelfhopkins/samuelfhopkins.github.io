\documentclass[11pt]{article}
\usepackage[top=1in, bottom=1in, left=1in, right=1in]{geometry}

\usepackage{amsmath}
\usepackage{amssymb}

\title{Math 211 (Modern Algebra II), HW\# 5, \\ {\normalsize Spring 2025; Instructor: Sam Hopkins; Due: Wednesday, April 2nd}}
\date{}

\begin{document}

\maketitle

\thispagestyle{empty}

\vspace{-1cm}

\begin{enumerate}

\item Let $K$ be a field and consider $K(x)$, the field of rational functions in the variable $x$, as a (simple, transcendental) extension of $K$. On a previous homework, you found some properties of the Galois group $\mathrm{Aut}_K(K(x))$. In this problem, you will fully describe $\mathrm{Aut}_K(K(x))$.
\begin{enumerate}
    \item For a rational function $0 \neq f/g \in K(x)$ with $f,g \in K[x]$ relatively prime, define its \emph{degree} to be $\mathrm{deg}(f/g) := \mathrm{max}(\mathrm{deg}(f),\mathrm{deg}(g))$. Show that $[K(x) : K(f/g)] = \mathrm{deg}(f/g)$ if $\mathrm{deg}(f/g) \geq 1$. {\bf Hint}: $x$ is a root of the polynomial $\varphi(y) = (f/g)g(y)-f(y) \in K(f/g)[y]$; you may use without proof the fact that this polynomial is irreducible.
    \item Let $f/g \in K(x)$ with $\mathrm{deg}(f/g) \geq 1$. Explain why the assignment $\sigma\colon x \mapsto f/g$ induces a homomorphism $\sigma \colon K(x) \to K(x)$, which is an automorphism if and only if $\mathrm{deg}(f/g)=1$.
    \item Conclude that $\mathrm{Aut}_K(K(x))$ consists exactly of the assignments $x \mapsto (ax+b)/(cx+d)$ with $a,b,c,d \in K$ and $ad - bc \neq 0$. (These are called \emph{fractional linear transformations}, and can be viewed as invertible $2\times 2$ matrices with entries in $K$.)
\end{enumerate}

\item Let $K$ be a field, $L/K$ an extension, and $S \subseteq L$ a subset that is algebraically independent over $K$. Let $u, v \in L$ with $v \in S$ and $u \notin S$. Suppose that $u$ is algebraic over $K(S)$ but that $u$ is not algebraic over $K(S \setminus \{v\})$. Show that $v$ is algebraic over $K((S \setminus \{v\}) \cup \{u\})$. (This is called the \emph{exchange lemma} for transcendence bases.)

\item In this problem, you will explore $\mathrm{Aut}_{\mathbb{Q}}(\mathbb{C})$, the field automorphisms of the complex numbers. We already know that two elements of $\mathrm{Aut}_{\mathbb{Q}}(\mathbb{C})$ are the identity and complex conjugation $a+bi \mapsto a-bi$. You will show that there are many other ``wild'' elements.
\begin{enumerate}
\item Show that a transcendence basis of $\mathbb{C}$ over $\mathbb{Q}$ is infinite. {\bf Hint}: First, note $\mathbb{Q}(x_1,\ldots,x_n)$ is countable for any finite $n \geq 1$ (why?). Then you may use the fact (that we did not prove in class but which is in the book) that if $K$ is an infinite field, the algebraic closure $\overline{K}$ of $K$ has the same cardinality as $K$. But $\mathbb{C}$ is uncountable!
\item Let $S$ be a transcendence basis of $\mathbb{C}$ over $\mathbb{Q}$. Show that any permutation of $S$ induces an automorphism in $\mathrm{Aut}_{\mathbb{Q}}(\mathbb{C})$. {\bf Hint}: First, observe in general that if the set $S$ is algebraically independent over the field $K$, then any permutation of $S$ induces an automorphism of $K(S)$. Then you may use the fact (that we did not prove in class but which is in the book) that if $K_1$ and $K_2$ are fields and $L_1/K_1$ and $L_2/K_2$ are algebraic closures, for any isomorphism $\varphi\colon K_1 \to K_2$ there is an isomorphism $\varphi \colon L_1 \to L_2$ extending it.
\item Conclude that $\mathrm{Aut}_{\mathbb{Q}}(\mathbb{C})$ is infinite.
\end{enumerate}
(The only automorphisms in $\mathrm{Aut}_{\mathbb{Q}}(\mathbb{C})$ that are \emph{continuous} with respect to the standard topology on $\mathbb{C}$ are the identity and complex conjugation. The other ``wild'' automorphisms are very wild indeed - their existence depends on the axiom of choice!)

\end{enumerate}


\end{document}