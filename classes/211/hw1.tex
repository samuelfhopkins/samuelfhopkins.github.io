\documentclass[11pt]{article}
\usepackage[top=1in, bottom=1in, left=1in, right=1in]{geometry}

\usepackage{amsmath}
\usepackage{amssymb}

\title{Math 211 (Modern Algebra II), HW\# 1, \\ {\normalsize Spring 2026; Instructor: Sam Hopkins; Due: Friday, January 30th}}
\date{}

\begin{document}

\maketitle

\thispagestyle{empty}

\begin{enumerate}

\item Let $K$ be a field and $L/K$ a finite extension. Recall $[L:K]$ denotes the degree of $L$ over $K$. Prove the following:
\begin{enumerate}
\item $[L:K]=1$ if and only if $L=K$.
\item If $[L:K]$ is a prime number, then there are no intermediate fields between $K$ and $L$.
\item If $u \in L$ is an algebraic element of degree $n$ over $K$, then $n$ divides $[L:K]$.
\end{enumerate}

\item Let $K$ be a field. Recall that $K[x]$ denotes the polynomial ring, and $K(x)$ denotes the field of rational functions, both with coefficients in $K$. We have seen that a basis of $K[x]$ as a $K$-vector space is $\{x^j\colon j \geq 0\}$. Prove that the following is a $K$-basis of $K(x)$:
\[ \{ x^j\colon j\geq 0\} \cup\left \{\frac{x^j}{P(x)^k}\colon k \geq 1, \, P(x) \in K[x] \textrm{ monic and irreducible}, \, 0 \leq j < \mathrm{deg}(P(x)) \right \} .\]
{\bf Hint}: remember the partial fraction decomposition of a rational function.

\item Let $f(x)=x^3-2x+2 \in \mathbb{Q}[x]$, a polynomial which is irreducible over the rational numbers. (Look up ``Eisenstein's criterion'' if you want to see why it is irreducible.) In fact, $f(x)$ has a unique real root, call it $u \in \mathbb{R}$. Let $L=\mathbb{Q}(u)$. We have seen that $\{1,u,u^2\}$ is a $\mathbb{Q}$-basis of $L$.
\begin{enumerate}
\item Express $u^4-2u^3+u^2-4 \in L$ as a $\mathbb{Q}$-linear combination of $\{1,u,u^2\}$.
\item Express $(u^2-3u+1)^{-1} \in L$ as a $\mathbb{Q}$-linear combination of $\{1,u,u^2\}$.
\end{enumerate}
{\bf Hint}: like we saw in class, use polynomial long division and the Euclidean gcd algorithm.

\item Let $L=\mathbb{Q}(\sqrt{2},\sqrt{3})$. Find $[L:\mathbb{Q}]$ and find a $\mathbb{Q}$-basis of $L$.

\item Recall that a real number $c \in \mathbb{R}$ is called \emph{constructible} if we can produce the point $(0,c) \in \mathbb{R}^2$ starting from the integer lattice $\mathbb{Z}^2 \subseteq \mathbb{R}^2$ and using a straightedge and compass. Prove that if $c \geq 0$ is constructible, then $\sqrt{c}$ is constructible.

\end{enumerate}


\end{document}
