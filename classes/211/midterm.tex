\documentclass[11pt]{article}
\usepackage[top=1in, bottom=1in, left=1in, right=1in]{geometry}

\usepackage{amsmath}
\usepackage{amssymb}

\title{Math 211 (Modern Algebra II), Midterm Exam, \\ {\normalsize Spring 2025; Instructor: Sam Hopkins; Taken on: Wednesday, February 26th}}
\date{}

\begin{document}

\maketitle

Each problem is worth 10 points, for a total of 50 points. You have 80 minutes to do the exam. Partial credit will be given generously, so write as much as you know for each problem.

\thispagestyle{empty}
\begin{enumerate}

\item Let $f(x) = x^2 - x - 1 \in \mathbb{Q}[x]$, a polynomial which is irreducible over $\mathbb{Q}$, and let $u$ be the unique positive real root of $f(x)$. Consider $L=\mathbb{Q}(u)$ as extension of $K=\mathbb{Q}$.
\begin{enumerate}
\item What is the degree $[L:K]$? 
\item Write a basis of $L$ over $K$.
\item Express $u^2+u+1$ in your basis.
\item Express $u^{-1}$ in your basis.
\end{enumerate}

\item \begin{enumerate}
\item Give a specific example of a field extension which is finitely generated but not algebraic. Explain why your example works.
\item Give a specific example of a finite extension of $\mathbb{Q}$ which is not Galois. Explain why your example works.
\end{enumerate}

\item Give a specific example of a finite extension of fields $L/K$ and a subgroup $H \subseteq \mathrm{Aut}_{K}(L)$ of the Galois group whose fixed field $H' = \{u \in L\colon \sigma(u) = u \textrm{ for all } \sigma \in H\}$ is neither $K$ nor $L$. Explain why your example works.

\item Does the polynomial $f(x) = x^3+4x \in \mathbb{R}[x]$ split over the real numbers $\mathbb{R}$? Explain what this means and why or why not. If it does not split over $\mathbb{R}$, then what is the smallest extension of $\mathbb{R}$ where it does split? Show \emph{how} it splits, either in $\mathbb{R}$ or in the extension.

\item An element $x\in K$ of a field $K$ is called a \emph{square} if $x=y^2$ for some $y\in K$. Let $K$ be a finite field of characteristic $2$. Prove that every element of $K$ is a square. {\bf Hint}: recall the Frobenius automorphism.

\end{enumerate}


\end{document}
