\documentclass[11pt]{article}
\usepackage[top=1in, bottom=1in, left=1in, right=1in]{geometry}

\usepackage{amsmath}
\usepackage{amssymb}

\title{Math 211 (Modern Algebra II), HW\# 2, \\ {\normalsize Spring 2025; Instructor: Sam Hopkins; Due: Monday, February 10th}}
\date{}

\begin{document}

\maketitle

\thispagestyle{empty}

\begin{enumerate}

\item Let $K$ be a field and let $L=K(x)$ be the field of rational functions with coefficients in $K$. Consider the Galois group $\mathrm{Aut}_{K}(L)$.
\begin{enumerate}
\item For $a \in L$ with $a \neq 0$, define $\sigma_a \colon L \to L$ by $\sigma_a( f(x) / g(x) ) = f(ax) / g(ax)$. Show that~$\sigma_a \in \mathrm{Aut}_{K}(L)$. Conclude that if $K$ is infinite, then $\mathrm{Aut}_{K}(L)$ is infinite.
\item For $b \in L$, define $\tau_b\colon L \to L$ by $\tau_b(f(x)/g(x)) = f(x+b) / g(x+b)$. Show~$\tau_b \in \mathrm{Aut}_{K}(L)$. Show that if $a \neq 1$ and $b \neq 0$, then $\sigma_a \tau_b \neq \tau_b \sigma_a$. Conclude that $\mathrm{Aut}_{K}(L)$ is nonabelian.
\end{enumerate}

\item Let $L=\mathbb{R}$, the real numbers, viewed as an extension of $K=\mathbb{Q}$, the rational numbers. Consider the Galois group $\mathrm{Aut}_{K}(L)$.
\begin{enumerate}
\item Let $\sigma \in \mathrm{Aut}_{K}(L)$. Prove that $u \geq 0$ if and only if $\sigma(u) \geq 0$. Conclude that $\sigma$ preserves the order on $\mathbb{R}$. {\bf Hint:} the nonnegative numbers in $\mathbb{R}$ are exactly those which are squares.
\item Use part (a) to show that $\mathrm{Aut}_{K}(L)$ is trivial. {\bf Hint:} every real number can be ``trapped'' between two rational numbers that are arbitrarily close to it.
\end{enumerate}

\item Let $L = \mathbb{Q}(\omega, \, \sqrt[3]{2})$, viewed as an extension of $K=\mathbb{Q}$, where $\omega = e^{2\pi i / 3} = \frac{-1+\sqrt{-3}}{2}$ is a primitive cube root of unity.  Notice that the roots of $f(x)=x^3-2$ are $\sqrt[3]{2}$, $\omega\sqrt[3]{2}$, and $\omega^2\sqrt[3]{2}$, so~$L$ is the field we get by adjoining all roots of $f(x)$ to $\mathbb{Q}$ (i.e., $L$ is the \emph{splitting field} of $f(x)$).

\begin{enumerate}
\item What is the degree $[L:K]$? Find a $K$-basis of $L$. {\bf Hint}: looking ahead to the other parts can help you answer this one.
\item Prove that $L/K$ is a Galois extension.
\item Prove that the Galois group $\mathrm{Aut}_{K}(L)$ is isomorphic to the symmetric group $S_3$.
\item Draw the subgroup structure of $S_3$ and the subfield structure of $L$ and show how they match up according to the Fundamental Theorem of Galois Theory.
\item Which subgroups of $S_3$ are normal? Which subfields of $L$ are Galois over $K$? How do these correspond?
\end{enumerate}

\end{enumerate}


\end{document}
