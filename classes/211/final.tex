\documentclass[11pt]{article}
\usepackage[top=1in, bottom=1in, left=1in, right=1in]{geometry}

\usepackage{amsmath}
\usepackage{amssymb}

\title{Math 211 (Modern Algebra II), Final Project, \\ {\normalsize Spring 2025; Instructor: Sam Hopkins; Presentations on: April 21st; Reports due: April 23rd}}
\date{}

\begin{document}

\maketitle

\thispagestyle{empty}

\vspace{-1.5cm}

For your final project, you will do independent research on a topic in algebra of your choosing, and then {\bf give a short presentation} and {\bf write a short report} about your topic. The presentations should be about 20 minutes long, with an additional 5 minutes for questions, and will take place in class on Monday, April 21st. The reports should be about 4 pages, typed (preferably in LaTeX), with references, and are due on Wednesday, April 23rd, the last day of class. For topics, you can pick one from the following list, or can come up with something else on your own. But please let me know what topic you choose, so that we can have a variety of topics among all students and so that I can suggest some sources.

Possible topics for your final project:

\vspace{-0.2cm}


\begin{enumerate}

\item {\bf Representation theory of finite groups.} ``Representation theory'' is a basic idea in algebra where we ``represent'' some abstract algebraic structure in a more ``concrete'' way. (We had a guest lecture on this.) For groups, this means representing an abstract group as a group of \emph{matrices}. You could choose this topic if you enjoyed our study of groups. Some results you could focus on for your project are the facts that, over the complex numbers, any finite-dimensional representation of a finite group is a direct sum of irreducible representations, and that for each such group, there are finitely many irreducible representations.

\item {\bf Basic commutative algebra.} Algebraic geometry studies algebraic varieties, spaces defined by polynomial equations, like the way $x^2+y^2=1$ defines a circle. Commutative algebra, the study of modules over polynomial rings, is the basic algebra underlying algebraic geometry. You could choose this topic if you liked our study of rings. A possible result you could focus on for your project is Hilbert's Nullstellensatz, which gives a precise relationship between an algebraic variety and the commutative ring of polynomial functions on that variety.

\item {\bf Further topics in field theory \& Galois theory.} If you enjoyed what we studied most recently, field theory and Galois theory, there are many further topics in this area that you could do a project on. For example, you could learn about the inverse Galois problem, which asks which finite groups are realizable as Galois groups of finite extensions of the rational numbers. Or you could explore origami (paper folding): we saw how compass and straightedge constructions allow one to solve quadratic equations; origami allows one to solve cubic equations as well.

\item {\bf Basic algebraic number theory.} Arithmetic properties of numbers, such as their being prime or more generally their prime factorizations, repeatedly came up in our study of basic algebraic structures like groups and rings. Conversely, number theoretic questions can be attacked with algebra. Your project could be on a topic in algebraic number theory: e.g., the rings of integers of number fields, or the $p$-adic numbers (which we had a guest lecture on).

\item {\bf Category theory.} We saw many times the idea that to study a kind of algebraic structure (like groups, rings, or modules), it is useful to consider the collection of all these structures together with the maps (homomorphisms) between them. Category theory is an abstraction of this idea, which allows one to compare and transfer results in different algebraic settings. You could do a project on basic category theoretic definitions and examples of categories.

\end{enumerate}

\end{document}
