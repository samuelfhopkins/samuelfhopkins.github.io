\documentclass[11pt]{article}
\usepackage[top=1in, bottom=1in, left=1in, right=1in]{geometry}

\usepackage{amsmath}
\usepackage{amssymb}

\title{Math 211 (Modern Algebra II), HW\# 3, \\ {\normalsize Spring 2025; Instructor: Sam Hopkins; Due: Wednesday, February 19th}}
\date{}

\begin{document}

\maketitle

\thispagestyle{empty}

Throughout, recall that for a prime power $q=p^n$, $\mathbb{F}_q$ denotes the field with $q$ elements, which we proved in class exists and is unique. Also, there are only four questions this week.

\begin{enumerate}

\item In this problem, you will construct $\mathbb{F}_4$ ``from scratch.'' Let us call the elements of $\mathbb{F}_4$ $\{0,1,a,b\}$. Since the characteristic of $\mathbb{F}_4$ is $2$, we know how $0$ and $1$ must add and multiply. So what we need to figure out is how $a$ and $b$ behave.
\begin{enumerate}
\item Write down the addition table of $\mathbb{F}_4$. {\bf Hint}: remember that addition is commutative, that the characteristic of $\mathbb{F}_4$ is $2$, and that additive inverses have to exist and be unique.
\item Write down the multiplication table of $\mathbb{F}_4$. {\bf Hint}: remember that multiplication is commutative, and that multiplicative inverses have to exist and be unique.
\item Consider the map $\varphi\colon \mathbb{F}_4 \to \mathbb{F}_4$ given by $\varphi\colon x \mapsto x^2$. Explain, using your tables, why this $\varphi$ is an automorphism. What are the fixed points of $\varphi$?
\end{enumerate}

\item Let $p$ be a prime. Recall that for a finite field $K$ of characteristic $p$, the \emph{Frobenius automorphism} $\varphi\colon K \to K$ is given by $\varphi\colon x \mapsto x^p$.

\emph{Fermat's Little Theorem} says that $a^p \equiv a \mod p$ for all integers $a\in\mathbb{Z}$. On a homework assignment from last semester you proved Fermat's Little Theorem using some group theory. Give another proof of Fermat's Little Theorem by using the Frobenius automorphism. \\{\bf Hint}: how must $\varphi$ behave on $\mathbb{F}_p$ itself?

\item Let $p$ be a prime and let $f(x) \in \mathbb{F}_p[x]$ be irreducible of degree $n$. Let $g(x) = x^{p^n}-x\in \mathbb{F}_p[x]$. Prove that $f(x)$ divides $g(x)$. {\bf Hint}: recall that $\mathbb{F}_{p^n}$ is the splitting field of $g(x)$.

\item Let $K=\mathbb{F}_2(t)$ be the field of rational functions, in the variable $t$, with coefficients in~$\mathbb{F}_2$. (We use $t$ because we also want to consider polynomials, in the usual variable $x$, over this field.) Consider the polynomial $f(x) = x^2 - t \in K[x]$.
\begin{enumerate}
\item Explain why $f(x)$ is irreducible.
\item Explain why $f(x)$ is not separable. {\bf Hint}: recall the relationship we discussed in class between the separability of a polynomial and its formal derivative.
\end{enumerate}
(This is the simplest example of a polynomial which is irreducible but not separable.)
\end{enumerate}


\end{document}
