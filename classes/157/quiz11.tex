\documentclass[11pt]{article}
\usepackage[top=1in, bottom=1in, left=1in, right=1in]{geometry}

\usepackage{amsmath}
\usepackage{amssymb}
\usepackage{graphicx}

\title{Quiz \#11, 4/18\\ Math 157 (Calculus II), Spring 2025}
\date{}

\begin{document}

\maketitle

\thispagestyle{empty}

\vspace{-2cm}

Problem 1 is worth 4 points, and Problem 2 is worth 6 points, for a total of 10 points. Remember to \emph{show your work} on all problems!

\begin{enumerate}

\item For each of the following series, decide if it converges or diverges. Explain your answer.

\begin{enumerate}
\item $\displaystyle \sum_{n=1}^{\infty} (-1)^{n-1} \cdot \frac{1}{\sqrt{n}}$ \hfill ({\bf Hint:} it's an alternating series.)
\item $\displaystyle \sum_{n=1}^{\infty} (-1)^{n-1}\cdot \frac{n^2-n+1}{3n^2+n-2} $ \hfill ({\bf Hint:} an alternating series, but look at limit of terms.)
\item $\displaystyle \sum_{n=1}^{\infty} \frac{3^n-10}{2^n+10}$ \hfill ({\bf Hint:} use the ratio test, or look at limit of the terms.)
\item $\displaystyle \sum_{n=1}^{\infty} \frac{4n}{3^n}$ \hfill ({\bf Hint:} use the ratio test.)
\end{enumerate}

\vspace{1.75in}

\item Consider the rational function $f(x) = \displaystyle \frac{1}{1+2x}$.
\begin{enumerate}
\item Express this function as a power series centered at zero: $f(x) = \displaystyle \sum_{n=0}^{\infty} c_n x^n$.
\item Determine the radius of convergence $R$ of the power series you found in part (a).
\end{enumerate}

\end{enumerate}

\end{document}
