\documentclass[10pt]{article}  
% use "amsart" instead of "article" for AMSLaTeX format
\usepackage{geometry} 
% See geometry.pdf to learn the layout options. There are lots.
\geometry{letterpaper}
% ... or a4paper or a5paper or ... 
%\geometry{landscape}
% Activate for rotated page geometry
%\usepackage[parfill]{parskip}
% Activate to begin paragraphs with an empty line rather than an indent
\usepackage{graphicx}
% Use pdf, png, jpg, or eps§ with pdflatex; use eps in DVI mode
% TeX will automatically convert eps --> pdf in pdflatex	
														
\usepackage{fullpage}
\usepackage{amsmath}
\usepackage{amsfonts}
\usepackage{amssymb}

%SetFonts

\usepackage[inline]{enumitem}

\usepackage{multicol}

\title{Math 157 -- Calculus II Final Exam -- Spring 2024}
%\author{The Author}
\date{April 30, 2024}	% Activate to display a given date or no date

\begin{document}
\maketitle
%\section{}
%\subsection{}

\begin{center}
\noindent \textbf{SHOW ALL WORK. Justify your answers! Simplify your answers. Give step-by-step explanations to get credit for answers. Give EXACT answers whenever possible.\\Solve all parts of any 10 out of the 15 problems below.\\Each of the 15 problems = 20 points. Exam total = 200 points.}
\end{center}

\begin{enumerate}

\item 
\begin{enumerate}

	\item Find the area bounded by the curves $y=x^2-1$ and $y=2x+7$.
	
	\item Find the average value $f_{\mathrm{avg}}$ of $f(x)=\dfrac{1}{x^2}$ on the interval $[1,3]$ and $c$ in the given interval such that $f_{\mathrm{avg}}=f(c)$.
	
\end{enumerate}

\item Let $R$ be the region in the first quadrant below the curve $y=\sqrt{x}$ from $x=1$ to $x=2$. Compute the volume of the solid obtained by rotating $R$:
\begin{multicols}{2}
\begin{enumerate}

\item about the $x$-axis;

\item about the $y$-axis.

\end{enumerate}
\end{multicols}

\item Evaluate the integrals:

\begin{multicols}{2}
\begin{enumerate}

\item $\displaystyle{\int_{0}^{2\pi}x^2 \sin(2x)\,dx}$\,;

\item $\displaystyle \int e^{3x} \cos x \; dx$\,.

\end{enumerate}
\end{multicols}

\item Determine whether the integral is convergent or divergent. Evaluate the integrals that are convergent.

\begin{multicols}{2}
\begin{enumerate}

\item $ \displaystyle\int_{1}^{\infty}\frac{x}{e^{2x}}\,dx$\,;
 
\item $\displaystyle{\int_{3}^{\infty}\dfrac{1}{(x-2)^{3/2}}\,dx}$\,.

\end{enumerate}
\end{multicols}

\item Evaluate the integrals:

\begin{multicols}{2}
\begin{enumerate}

\item  $\displaystyle\int \sec x \tan^3 x \,dx$\,;

\item  $\displaystyle\int \sin^2 x \cos^3 x\, dx$\,.

\end{enumerate}
\end{multicols}

\item Evaluate the integrals:

\begin{multicols}{2}
\begin{enumerate}

\item $\displaystyle\int\frac{1}{(x^2+9)^{3/2}}\,dx$\,;

\item $\displaystyle \int \frac{2x+3}{x^2-4} \; dx$\,.

\end{enumerate}
\end{multicols}

\item 
\begin{enumerate}

\item Write the partial fractions decomposition of $\dfrac{10}{(x-1)(x^2+9)}$\,.

\item Evaluate the integral $\displaystyle\int\frac{10}{(x-1)(x^2+9)}\,dx$\,.

\end{enumerate}


\begin{center}
\textbf{(continued on the next page)}
\end{center}

\newpage

\begin{center}
\textbf{(continued from the previous page)}
\end{center}

%\item Determine if each sequence $(a_n)$ converges or diverges, and find the limit if it converges:
%
%\begin{multicols}{2}
%\begin{enumerate}
%
%\item $a_n=\dfrac{\cos n}{\ln n}$, \; $n \ge 2$\,;
%
%\item $a_n=\dfrac{n!}{3^n}$, \; $n \ge 1$\,.
%
%\end{enumerate}
%\end{multicols}
%
%\item Determine if each series converges or diverges, and find the limit if it converges:
%
%\begin{multicols}{2}
%\begin{enumerate}
%
%\item $\displaystyle \sum _{n=1} ^{\infty} \ln\left(1+\frac{1}{n}\right)$\,;
%
%\item $\displaystyle \sum_{n=1}^{\infty}\frac{2^n+(-1)^n}{5^n}$\,.
%
%\end{enumerate}
%\end{multicols}

%\item Determine whether each of the following sequences $(a_n)_{n\ge 1}$ converges, and if so, find its limit. Remember to justify your answers.
%
%\begin{multicols}{2}
%\begin{enumerate}
%
%\item $\displaystyle a_n=\cos\left(\frac{1}{n^n}\right)$\,;
%
%\item $\displaystyle a_n=ne^{1/n}$\,;
%
%\item $\displaystyle a_n=\frac{\sin n}{\ln(n+1)}$\,;
%
%\item $\displaystyle a_n=n^2\ln\left(1+\frac{1}{n}\right)$\,.
%
%\end{enumerate}
%\end{multicols}


\item
\begin{enumerate}

\item Find an equation of the line tangent to the curve given by $x=2+\ln t$, $y=t^2-3$ at the point $(2,-2)$.

\item Find the length of the curve defined by $x=-\sin^3 t, y=-\cos^3 t$\ over the interval $0\le t\le \dfrac{\pi}{2}$.

\end{enumerate}

\item Compute the surface area of the surface obtained by rotating the curve given by $y=x^3$ from $x=0$ to $x=1$ about the $x$-axis.

\item Determine whether each of the following series converges conditionally, converges absolutely, or diverges. Remember to justify your answers.

\begin{multicols}{2}
\begin{enumerate}

\item $\displaystyle \sum_{n=1}^{\infty}\frac{2^n n^3}{n!}$\,;

\item $\displaystyle \sum_{n=1}^{\infty} \frac{2n^2+3n-2}{3n^2+5n+1}$\,;

\item $\displaystyle \sum_{n=1}^{\infty} \frac{(-1)^{n}}{2n+1}$\,;

\item $\displaystyle{\sum_{n=1}^{\infty} \dfrac{(-1)^n\arctan n}{n^2}}$\,.

\end{enumerate}
\end{multicols}

\item
\begin{enumerate}

\item Graph the curve $r=2(1+\cos \theta)$.

\item Find the area of the region in the plane enclosed by the curve $r=2(1+\cos \theta)$.

\end{enumerate}


\item Find the interval and radius of convergence of the power series $\displaystyle{\sum_{n=1}^{\infty}\dfrac{(5x-4)^n}{n^3}}$\,.


\item A spring has a natural length of 40 cm. If a 60 N force is needed to keep the spring compressed 10 cm,
\begin{enumerate}

\item how much work is done during this compression?

\item how much work is required to compress the spring to a length of 25 cm?

\end{enumerate}

\emph{Hint:} Recall that Hooke's Law says that the force needed to keep a spring compressed a distance $x$ beyond its natural length is $kx$, where $k$ is the spring constant of the spring.

\item Consider the function $f(x) = \sin(x)$.
\begin{enumerate}

\item Write the degree three Taylor polynomial $T_3(x)$, centered at $x=0$, for this $f(x)$. 

\item Use your answer in part (a) to give an estimate for the value of $f(-1)$. 

\item Give an upper bound on the error for your estimate from part (b). \emph{Hint:} Recall that the Taylor series for $\sin x$ at $x=0$ is alternating.

%\emph{Hint}: Recall that the remainder $R_n(x) = f(x) - T_n(x)$ satisfies $|R_n(x)| \le \dfrac{M}{(n+1)!}|x|^{n+1}$ for $|x|\le d$, where $M$ is the maximum value of $|f^{(n+1)}(x)|$ attained on this interval.

\end{enumerate}

\item
\begin{enumerate}

\item Approximate $\displaystyle \int_{-2}^{4} (x+1)^2 \; dx$ by using the midpoint rule with $n=3$ subintervals. 

\item What is the error of your approximation compared to the true value of this definite integral?

\end{enumerate}

\end{enumerate}

\end{document}  