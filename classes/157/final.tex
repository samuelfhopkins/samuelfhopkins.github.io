\documentclass[10pt]{article}  
% use "amsart" instead of "article" for AMSLaTeX format
\usepackage{geometry} 
% See geometry.pdf to learn the layout options. There are lots.
\geometry{letterpaper}
% ... or a4paper or a5paper or ... 
%\geometry{landscape}
% Activate for rotated page geometry
%\usepackage[parfill]{parskip}
% Activate to begin paragraphs with an empty line rather than an indent
\usepackage{graphicx}
% Use pdf, png, jpg, or eps§ with pdflatex; use eps in DVI mode
% TeX will automatically convert eps --> pdf in pdflatex	
														
\usepackage{fullpage}
\usepackage{amsmath}
\usepackage{amsfonts}
\usepackage{amssymb}

%SetFonts

\usepackage[inline]{enumitem}

\usepackage{multicol}

\title{Math 157 -- Calculus II Final Exam -- Spring 2023}
%\author{The Author}
\date{May 2, 2023}	% Activate to display a given date or no date

\begin{document}
\maketitle
%\section{}
%\subsection{}

\begin{center}
\noindent \textbf{SHOW ALL WORK. Justify your answers! Simplify your answers.\\Show details to earn full points. Give EXACT answers whenever possible.\\ Solve all parts of any 10 out of the 15 problems below.\\Each of the 15 problems = 10 points. Exam total = 100 points.}
\end{center}

\begin{enumerate}

\item 
\begin{enumerate}
\item Sketch the region enclosed by the curves $y=x^3$ and $y=x$ and find its area.

\item Find the average value $f_{\mathrm{avg}}$ of the function $f(x)=x^2$ on the interval $[-1,2]$.

\end{enumerate}

\item Let $R$ be the region in the first quadrant below the curve $y=x^3$ from $x=1$ to $x=2$. Compute the volume of the solid obtained by rotating $R$:
\begin{multicols}{2}
\begin{enumerate}

\item about the $x$-axis;

\item about the $y$-axis.

\end{enumerate}
\end{multicols}

\item Evaluate the integrals:

\begin{multicols}{2}
\begin{enumerate}

\item $ \displaystyle\int\frac{\ln x}{x^2}\,dx$\,;

\item $\displaystyle\int\frac{x}{e^{4x}}\,dx$\,.

\end{enumerate}
\end{multicols}

\item Evaluate the integrals:

\begin{multicols}{2}
\begin{enumerate}

\item $ \displaystyle\int_0^{\infty}\frac{x}{(x^2+1)^3}\,dx$\,;
 
\item $\displaystyle\int_{0}^{\pi/4}{\sin^2x\,\cos^3x\,dx}$\,.

\end{enumerate}
\end{multicols}

\item Evaluate the integrals:

\begin{multicols}{2}
\begin{enumerate}

\item  $\displaystyle\int \tan^5 x \sec^3 x\,dx$\,;

\item  $\displaystyle\int\frac{1}{\sqrt{4-x^2}}\,dx$\,.

\end{enumerate}
\end{multicols}

\item Evaluate the integrals:

\begin{multicols}{2}
\begin{enumerate}

\item $\displaystyle\int\frac{\sqrt{16-9x^2}}{x^2}\,dx$\,;

\item $ \displaystyle\int\frac{4x}{(x+1)(x^2+1)}\,dx$\,.

\end{enumerate}
\end{multicols}

\item Compute the area of the surface obtained by rotating the curve given by $y=2\,\sqrt{x}$ for $x=0$ to $x=1$ about the $x$-axis.

\item
\begin{enumerate}

\item Find an equation of the line tangent to the curve given by $x=2+\ln t$, $y=t^2-3$\ at the point $(2,-2)$.

\item Find the length of the curve defined by $x=-\sin^3 t, y=-\cos^3 t$\ over the interval $0\le t\le \frac{\pi}{2}$.

\end{enumerate}


\begin{center}
\textbf{(continued on the next page)}
\end{center}

\newpage

\begin{center}
\textbf{(continued from the previous page)}
\end{center}

%\item Determine if each sequence $(a_n)$ converges or diverges, and find the limit if it converges:
%
%\begin{multicols}{2}
%\begin{enumerate}
%
%\item $a_n=\dfrac{\cos n}{\ln n}$, \; $n \ge 2$\,;
%
%\item $a_n=\dfrac{n!}{3^n}$, \; $n \ge 1$\,.
%
%\end{enumerate}
%\end{multicols}
%
%\item Determine if each series converges or diverges, and find the limit if it converges:
%
%\begin{multicols}{2}
%\begin{enumerate}
%
%\item $\displaystyle \sum _{n=1} ^{\infty} \ln\left(1+\frac{1}{n}\right)$\,;
%
%\item $\displaystyle \sum_{n=1}^{\infty}\frac{2^n+(-1)^n}{5^n}$\,.
%
%\end{enumerate}
%\end{multicols}

\item Determine whether each of the following sequences $(a_n)_{n\ge 1}$ converges, and if so, find its limit. Remember to justify your answers.

\begin{multicols}{2}
\begin{enumerate}

\item $\displaystyle a_n=\frac{\sin n}{n}$\,;

\item $\displaystyle a_n=\arctan(n^2-1)$\,;

\item $\displaystyle a_n=n\cos\left(\frac{1}{n}\right)$\,;

\item $\displaystyle a_n=\left(\left(1+\frac{1}{n}\right)^{\!n}\right)^{\!\!n}$\,.

\end{enumerate}
\end{multicols}

\item Determine whether each of the following series converges conditionally, converges absolutely, or diverges. Remember to justify your answers.

\begin{multicols}{2}
\begin{enumerate}

\item $\displaystyle \sum_{n=1}^{\infty}\frac{\cos\sqrt{n}}{n^3}$\,;

\item $\displaystyle \sum_{n=1}^{\infty} \frac{2n^2+1}{3n^2+10n+2}$\,;

\item $\displaystyle \sum_{n=1}^{\infty} \frac{(-1)^{n}}{n+1}$\,;

\item $\displaystyle \sum_{n=1}^{\infty}\frac{2^n n^3}{n!}$\,.

\end{enumerate}
\end{multicols}


\item

\begin{enumerate}

\item Plot the point whose polar coordinates $\left(2, \frac{\pi}{4}\right)$  are given. Then find two other pairs of polar coordinates of this point, one with $r>0$ and one with $r<0$.
  
\item Compute the area of one loop of the polar curve given by $r=4\cos3\theta$. \emph{Hint:} Start by determining the values of $\theta$ where the loop begins and ends.

\end{enumerate}


\item Find the interval of convergence of the series $\displaystyle\sum_{n=1}^{\infty}\frac{(2x-3)^n}{5^n\sqrt{n}}$\,.


\item A cable 30 meters long weighs 600 Newtons and hangs from the top of a building that is 40 meters above the ground. Find the work needed to pull the upper two thirds of the cable to the top of the building.

\item Consider the function $f(x) = \sin x$.
\begin{enumerate}
\item Write the degree three Taylor polynomial $T_3(x)$, centered at $x=0$, for this $f(x)$. 

\item Use your answer in part (a) to give an estimate for the value of $f(\frac{1}{2})$. 

\item Give an upper bound on the error for your estimate from part (b). \emph{Hint}: Recall that the Taylor series for $\sin x$ is alternating.

\end{enumerate}

\item The interval $[-1,5]$\ is partitioned into $n$\ subintervals $[x_{k-1}, x_k]$\ for $k=1,...,n$\ each of width $\Delta x=x_k-x_{k-1}$. Choose any $x_k^*$\ such that $x_{k-1}\le x_k^*\le x_k$. Let the function $f$\ be continuous over $[-1,5]$. Do the following. 
\begin{enumerate}
\item State the limit definition of $\displaystyle\int_{-1}^{5}{f(x)}\,dx$ .

\item For $n=3$, write the midpoint approximation for the integral in part (a) in terms of $f$.

\item For $f(x)=x^2+2$, use the expression in part (b) to estimate the value of the integral in part (a).

\item What is the error of your approximation compared to the true value of this definite integral?
\end{enumerate}

\end{enumerate}

\end{document}  