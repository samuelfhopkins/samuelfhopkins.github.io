\documentclass[11pt]{article}
\usepackage[top=1in, bottom=1in, left=1in, right=1in]{geometry}

\usepackage{amsmath}
\usepackage{amssymb}
\usepackage{graphicx}

\title{Quiz \#8, 3/21\\ Math 157 (Calculus II), Spring 2024}
\date{}

\begin{document}

\maketitle

\thispagestyle{empty}

\vspace{-2cm}

Problem 1 is worth 5 points, and Problem 2 is worth 5 points, for a total of 10 points. Remember to \emph{show your work} on all problems!

\begin{enumerate}
\item Consider the polar curve $r=1-\cos(\theta)$ for $0 \leq \theta \leq 2\pi$.
\begin{enumerate}
\item First, make a chart or a plot of $r$ as a function of $\theta$.
\item Then, using the chart/plot in part (a) as a guide, sketch the graph of this polar curve.
\end{enumerate}

\vspace{3.25in}

\item Consider the polar curve $r=\theta(\pi-\theta)$ for $0 \leq \theta \leq \pi$. Compute the area inside of this curve.

\end{enumerate}

\end{document}
