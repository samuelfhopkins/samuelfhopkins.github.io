\documentclass[11pt]{article}
\usepackage[top=1in, bottom=1in, left=1in, right=1in]{geometry}

\usepackage{amsmath}
\usepackage{amssymb}
\usepackage{graphicx}

\title{Final Exam Study Guide \\ Math 157 (Calculus II), Spring 2025}
\date{}

\begin{document}

\maketitle

\thispagestyle{empty}

\vspace{-2.4cm}

\begin{enumerate}
\item Geometric applications of integrals [\S6.1, 6.2, 6.3, 8.1, 8.2]
\begin{enumerate}
\item Area between curves [\S6.1]: area between $y=f(x)$ and $y=g(x)$ is $\int_{a}^{b} |f(x)-g(x)| \; dx$.
\item Volume of general solid [\S6.2]: if $A(x)=$ area of cross-section, then volume is $\int_a^b A(x) \;dx$.
\item Volume of solid of revolution [\S6.2, 6.3]: ``disks/washers'' $\&$ ``cylindrical shells'' methods.  \\For region below curve $y=f(x)$ from $x=a$ to $x=b$:
\begin{enumerate}
\item rotated around $x$-axis, ``disks method'' gives volume $=\int_a^b \pi \, f(x)^2 \; dx$;
\item rotated around $y$-axis, ``shells method'' gives volume $=\int_a^b 2 \pi \, f(x) \, x \; dx$.
\end{enumerate}
\item Arc lengths of curves [\S8.1]: length of $y=f(x)$ from $x=a$ to $x=b$ is $\int_{a}^{b} \sqrt{1+(f'(x))^2} \; dx$.
\item Area of surface of revolution [\S8.2]: 
\begin{enumerate}
\item for $y=f(x)$ from $x=a$ to $x=b$ rotated about $x$-axis, area is $\int_{a}^{b} 2\pi f(x) \, \sqrt{1+(f'(x))^2} \; dx$;
\item for $x=g(y)$ from $y=c$ to $y=d$ rotated about $x$-axis, area is $\int_{c}^{d} 2\pi y \, \sqrt{1+(g'(y))^2} \; dy$.
\end{enumerate}
\end{enumerate}

\item Other applications of integrals [\S6.4, 6.5]
\begin{enumerate}
\item Work [\S6.4]: if $F(x)=$ force as function of distance, then work done is $W=\int_{a}^{b} F(x) \; dx$.
\item Average of function [\S6.5]: the average of $f(x)$ from $x=a$ to $x=b$ is $\frac{1}{b-a}\int_a^b f(x) \; dx$.
\end{enumerate}

\item Techniques for computing integrals [\S7.1, 7.2, 7.3, 7.4, 7.5]
\begin{enumerate}
\item Integration by parts [\S7.1]: $\int u \; dv = uv - \int v \; du$; choose $u$ using ``LIATE'' rule
\item Trigonometric integrals [\S7.2]: for $\int \sin^n(x) \cos^m(x) \; dx$, use the Pythagorean identity $\sin^2(x) + \cos^2(x)=1$ to isolate single factor of $\cos(x) \; dx$ or $\sin(x) \; dx$, then do a $u$-sub.
\item Trigonometric substitution [\S7.3]: 
\begin{enumerate}
\item for $a^2-x^2$ $\Rightarrow$ sub $x = a\sin(\theta)$, $dx = a\cos(\theta) \, d\theta$, and use $1-\sin^2(\theta) = \cos^2(\theta)$;
\item for $a^2+x^2$ $\Rightarrow$ sub $x = a\tan(\theta)$, $dx = a\sec^2(\theta) \, d\theta$, and use $1+\tan^2(\theta) = \sec^2(\theta)$.
\end{enumerate}
\item Integrating rational functions by partial fractions [\S7.4]: find roots of denominator $Q(x)$ and solve system of equations to write $P(x)/Q(x) = A/(x-a) + B/(x-b) + ...  + Z/(x-z)$ and use $\int A/(x-a) \;dx = A \ln(x-a)$; for repeated roots do $A_1/(x-a) + A_2/(x-a)^2 + \cdots$.
\end{enumerate}

\item Other concepts related to integration [\S7.7, 7.8]
\begin{enumerate}
\item Approximating definite integrals [\S7.7]: two good approximations of $\int_{a}^{b} f(x) \, dx$ are
\begin{enumerate}
\item midpoint approximation $M_n = \sum_{i=1}^{n} f(\overline{x}_i) \Delta x$ where $\overline{x}_i = \frac{x_{i-1}+x_i}{2}$;
\item trapezoid approximation $T_n = \frac{\Delta x}{2} (f(x_0)+2f(x_1) + 2f(x_2) + \cdots + 2f(x_{n-1}) + f(x_n) )$.
\end{enumerate}
\item Improper integrals [\S7.8]: $\int_{a}^{\infty} f(x) \; dx = \lim_{t\to \infty} \int_a^t f(x) \; dx$, et cetera.
\end{enumerate}

\pagebreak
\thispagestyle{empty}

\item Parametrized curves [\S10.1, 10.2]
\begin{enumerate}
\item Curve of form $x=f(t)$ and $y=g(t)$ for some auxiliary variable $t$ (``time'') [\S10.1]
\item Slope of tangent [\S10.2] to curve given by chain rule: $\frac{dy}{dx} = \frac{dy/dt}{dx/dt} = \frac{g'(t)}{f'(t)}$ 
\item Arc length [\S10.2] is {\small $\int_{a}^{b} \sqrt{(\frac{dy}{dt})^2 + (\frac{dx}{dt})^2} \; dt = \int_{a}^{b} \sqrt{g'(t)^2 + f'(t)^2} \; dt$}
\end{enumerate}

\item Polar coordinates and polar curves [\S10.3, 10.4]
\begin{enumerate}
\item Cartesian vs.~polar [\S10.3]: $(x,y) = (r\cos \theta, r \sin \theta)$ and $(r,\theta) = (\sqrt{x^2+y^2}, \arctan (\frac{y}{x}))$
\item Area inside [\S10.4] polar curve $r=f(\theta)$ for $\alpha \leq \theta \leq \beta$ is $\int_{\alpha}^{\beta} \frac{1}{2} r^2 \; d\theta=\int_{\alpha}^{\beta} \frac{1}{2} f(\theta)^2 \; d\theta$ 
\item Slope of tangent [\S10.4] to polar curve $r=f(\theta)$ given by chain and product rules: 
{\small \[\frac{dy}{dx} = \frac{dy/d\theta}{dx/d\theta} = \frac{\frac{d}{d\theta}(r \sin\theta)}{\frac{d}{d\theta}(r \cos\theta)} = \frac{f(\theta)\, \cos \theta  + f'(\theta) \, \sin \theta}{f'(\theta)\cos\theta-f(\theta)\, \sin\theta }\]}

\vspace{-0.2cm}
\item Arc length [\S10.4] of polar curve $r=f(\theta)$ is {\small $\int_{\alpha}^{\beta} \sqrt{r^2 + (\frac{dr}{d\theta})^2} \; d\theta = \int_{\alpha}^{\beta} \sqrt{f(\theta)^2 + f'(\theta)^2} \; d\theta$ }
\end{enumerate}

\item Sequences and series [\S11.1, 11.2, 11.3, 11.4, 11.5, 11.6, 11.7]
\begin{enumerate}
\item Sequence $\{a_n\}_{n=1}^{\infty} = a_1,a_2,\ldots$ is list of numbers, $\displaystyle \lim_{n\to \infty} a_n$ defined like $\displaystyle \lim_{x \to \infty} f(x)$ [\S11.1]
\item Series $\sum_{n}^{\infty} a_n$ is ``infinite sum'' $a_1+a_2+\cdots$ of terms $a_n$; its value is $s=\lim_{n\to\infty} s_n$ where $s_n = a_1+a_2+\cdots+a_n$ is the $n$th partial sum [\S11.2]
\item Important series: geometric series [\S11.2] $\sum_{n=1}^{\infty} ar^{n-1}$ converges if and only if $|r|<1$ (and $=\frac{a}{1-r}$ if it converges); $p$-series [\S11.3] $\sum_{n=1}^{\infty}\frac{1}{n^p}$ converges if and only if $p > 1$ 
\item Many tests for convergence / divergence of series:
\begin{enumerate}
\item (Divergence test [\S11.2]) If $\lim_{n\to \infty} a_n \neq 0$, series $\sum_{n}^{\infty} a_n$ diverges.
\item (Integral test [\S11.3]) If $f(x)$ continuous, decreasing, and positive, with $a_n = f(n)$, then $\sum_{n}^{\infty} a_n$ converges if and only if $\int_{1}^{\infty} f(x) \; dx$ converges. In this case, have error bounds for remainder $R_n = s-s_n$: $\int_{n+1}^{\infty} f(x) \; dx \leq R_n \leq \int_{n}^{\infty} f(x) \; dx$.
\item (Comparison tests [\S11.4] for series w/ positive terms) If $\sum_{n}^{\infty} b_n$ converges \& $a_n \leq b_n$, then $\sum_{n}^{\infty} a_n$ converges. If $\sum_{n}^{\infty} b_n$ diverges \& $a_n \geq b_n$, then $\sum_{n}^{\infty} a_n$ diverges. If $\lim_{n\to\infty} a_n/b_n \neq 0, \pm\infty$, then $\sum_{n}^{\infty} a_n$ converges if and only if $\sum_{n}^{\infty} b_n$ converges.
\item (Alternating series test [\S11.5]) Alternating series $\sum_{n=1}^{\infty} (-1)^{n-1} b_n$ converges as long as $b_{n+1} \leq b_n$ and $\lim_{n\to\infty} b_n=0$. In this case, have error bound: $|R_n| \leq b_{n+1}$.
\item (Ratio test [\S11.6]) For series $\sum_{n=1}^{\infty} a_n$, let $L = \lim_{n\to \infty} \frac{|a_{n+1}|}{|a_n|}$. If $L < 1$, series converges. If $L > 1$ (including $\infty$), series diverges. If $L=1$, test is inconclusive.
\end{enumerate}
\end{enumerate}

\item Power series and Taylor series [\S11.8, 11.9, 11.10, 11.11]
\begin{enumerate}
\item The ratio test tells us that any power series $\sum_{n=0}^{\infty} c_n (x-a)^n$ has a radius of convergence $R$ such that it converges when $|x-a|<R$ and diverges when $|x-a|>R$ [\S11.8]
\item Power series representations of functions $f(x) = \sum_{n=0}^{\infty} c_n (x-a)^n$; getting a representation for one function from another via algebraic manipulations (like substitution) [\S11.9]
\item Differentiate, integrate, and multiply power series like they are polynomials [\S11.9, 11.10]
\item Taylor series of $f(x)$ at $x=a$ is $\sum_{n=0}^{\infty} \frac{f^{(n)}(a)}{n!} (x-a)^n$, where $f^{(n)}$ is $n$th derivative [\S11.10]
\item Important Taylor series [\S11.10]: $\frac{1}{1-x} = \sum_{n=0}^{\infty} x^n$ ($R=1$); \; $e^x = \sum_{n=0}^{\infty} \frac{x^n}{n!}$ ($R=\infty$); \; $\sin(x) = \sum_{n=0}^{\infty} \frac{(-1)^{n-1}x^{2n+1}}{(2n+1)!}$ ($R=\infty$); \; $\cos(x)=\sum_{n=0}^{\infty}\frac{(-1)^{n}x^{2n}}{(2n)!}$ ($R=\infty$)
\item Taylor polynomial $T_n(x)$: $n$th partial sum of series; $f(x) \approx T_n(x)$ if $x\approx a$ [\S11.10, 11.11]
\end{enumerate}

\end{enumerate}

\end{document}