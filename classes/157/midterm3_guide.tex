\documentclass[11pt]{article}
\usepackage[top=0.75in, bottom=0.75in, left=1in, right=1in]{geometry}

\usepackage{amsmath}
\usepackage{amssymb}
\usepackage{graphicx}

\title{Midterm \#3 Study Guide \\ Math 157 (Calculus II), Spring 2025}
\date{}

\begin{document}

\maketitle

\thispagestyle{empty}

\vspace{-2cm}

\begin{enumerate}
\item Sequences and series [\S11.1, 11.2, 11.3, 11.4, 11.5, 11.6, 11.7]
\begin{enumerate}
\item Sequence $\{a_n\}_{n=1}^{\infty} = a_1,a_2,\ldots$ is list of numbers, $\displaystyle \lim_{n\to \infty} a_n$ defined like $\displaystyle \lim_{x \to \infty} f(x)$ [\S11.1]
\item Series $\sum_{n}^{\infty} a_n$ is ``infinite sum'' $a_1+a_2+\cdots$ of terms $a_n$; its value is $s=\lim_{n\to\infty} s_n$ where $s_n = a_1+a_2+\cdots+a_n$ is the $n$th partial sum [\S11.2]
\item Important series: geometric series [\S11.2] $\sum_{n=1}^{\infty} ar^{n-1}$ converges if and only if $|r|<1$ (and $=\frac{a}{1-r}$ if it converges); $p$-series [\S11.3] $\sum_{n=1}^{\infty}\frac{1}{n^p}$ converges if and only if $p > 1$ 
\item Many tests for convergence / divergence of series:
\begin{enumerate}
\item (Divergence test [\S11.2]) If $\lim_{n\to \infty} a_n \neq 0$, series $\sum_{n}^{\infty} a_n$ diverges.
\item (Integral test [\S11.3]) If $f(x)$ continuous, decreasing, and positive, with $a_n = f(n)$, then $\sum_{n}^{\infty} a_n$ converges if and only if $\int_{1}^{\infty} f(x) \; dx$ converges. In this case, have error bounds for remainder $R_n = s-s_n$: $\int_{n+1}^{\infty} f(x) \; dx \leq R_n \leq \int_{n}^{\infty} f(x) \; dx$.
\item (Comparison tests [\S11.4] for series w/ positive terms) If $\sum_{n}^{\infty} b_n$ converges \& $a_n \leq b_n$, then $\sum_{n}^{\infty} a_n$ converges. If $\sum_{n}^{\infty} b_n$ diverges \& $a_n \geq b_n$, then $\sum_{n}^{\infty} a_n$ diverges. If $\lim_{n\to\infty} \frac{a_n}{b_n}$ exists and is $\neq 0$, then $\sum_{n}^{\infty} a_n$ converges if and only if $\sum_{n}^{\infty} b_n$ converges.
\item (Alternating series test [\S11.5]) Alternating series $\sum_{n=1}^{\infty} (-1)^{n-1} b_n$ converges as long as $b_{n+1} \leq b_n$ and $\lim_{n\to\infty} b_n=0$. In this case, have error bound: $|R_n| \leq b_{n+1}$.
\item (Ratio test [\S11.6]) For series $\sum_{n=1}^{\infty} a_n$, let $L = \lim_{n\to \infty} \frac{|a_{n+1}|}{|a_n|}$. If $L < 1$, series converges. If $L > 1$ (including $\infty$), series diverges. If $L=1$, test is inconclusive.
\end{enumerate}
\end{enumerate}

\item Power series and Taylor series [\S11.8, 11.9, 11.10, 11.11]
\begin{enumerate}
\item The ratio test tells us that any power series $\sum_{n=0}^{\infty} c_n (x-a)^n$ has a radius of convergence $R$ such that it converges when $|x-a|<R$ and diverges when $|x-a|>R$ [\S11.8]
\item Power series representations of functions $f(x) = \sum_{n=0}^{\infty} c_n (x-a)^n$; getting a representation for one function from another via algebraic manipulations (like substitution) [\S11.9]
\item Differentiate, integrate, and multiply power series like they are polynomials [\S11.9, 11.10]
\item Taylor series of $f(x)$ at $x=a$ is $\sum_{n=0}^{\infty} \frac{f^{(n)}(a)}{n!} (x-a)^n$, where $f^{(n)}$ is $n$th derivative [\S11.10]
\item Important Taylor series [\S11.10]: $\frac{1}{1-x} = \sum_{n=0}^{\infty} x^n$ ($R=1$); \; $e^x = \sum_{n=0}^{\infty} \frac{x^n}{n!}$ ($R=\infty$); \; $\sin(x) = \sum_{n=0}^{\infty} \frac{(-1)^{n-1}x^{2n+1}}{(2n+1)!}$ ($R=\infty$); \; $\cos(x)=\sum_{n=0}^{\infty}\frac{(-1)^{n}x^{2n}}{(2n)!}$ ($R=\infty$)
\item Taylor polynomial $T_n(x)$: $n$th partial sum of series; $f(x) \approx T_n(x)$ if $x\approx a$ [\S11.10, 11.11]
\end{enumerate}
\end{enumerate}

\end{document}