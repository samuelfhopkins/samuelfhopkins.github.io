\documentclass[11pt]{article}

\usepackage{amsmath}
\usepackage{amssymb}

\usepackage{graphicx}
\usepackage{tikz}

\usepackage{ytableau}

\begin{document}

\thispagestyle{empty}

\begin{center}
{\bf Math 4707: Intro to combinatorics and graph theory \\
Spring 2020, Instructor: Sam Hopkins \\
Final exam- Due Wednesday, May 5th}
\end{center}

{\bf Instructions}: There are 5 problems, worth 20 points each, totaling 100 points. This is an open book, open library, open notes, open web, take-home exam, but you are not allowed to to interact with anyone (including online forums) except for me, the instructor. As always, in order to earn points you need to carefully \emph{explain your answer}.

\begin{enumerate}
\item (20 points total; 10 points each) 
\begin{enumerate}
\item How many paths are there in the plane $\mathbb{R}^2$ going from $(0,0)$ to $(50,100)$ taking unit steps in either the north or east direction at each step?
\item How many such paths are there which avoid passing through any of the $3$ ``bad'' points
\[ (10,11), \; (20,42), \; (30, 85)?\]
\end{enumerate}

\item (20 points) Let $t_n$ denote the number of trees on $n$ vertices, considered up to isomorphism (i.e., $t_n$ is the number of ``unlabeled trees'' on $n$ vertices). For example, the first several values of $t_n$ are 
\[t_1=1, \; t_2=1, \; t_3=1, \; t_4=2, \; t_5=3, \; t_6=6,\ldots\]
Prove that $\displaystyle t_n \leq \binom{2(n-1)}{n-1}$. 

{\bf Hint:} Did we talk about the problem of counting unlabeled trees somewhere in the textbook and/or lectures?

\item Fix integers $m, n$ with $1 \leq m \leq n$. In this problem we consider simple bipartite graphs $G$ with bipartitions $(X,Y)$, where $\#X=m$ and $\#Y=n$.
\begin{enumerate}
\item (5 points) Show that there exists such a $G$ with $(m-1)n$ edges for which there is no matching $M$ in $G$ containing $m$ edges.
\item (15 points) Show that for every such $G$ with at least $(m-1)n + 1$ edges, there must be a matching $M$ in $G$ containing $m$ edges.
\end{enumerate}

\pagebreak

\item (20 points) Your friend hands you a convex polyhedron in $\mathbb{R}^3$ which has triangular and hexagonal faces (and no other kinds of faces), and for which every vertex belongs to three edges. How many triangular faces must this polyhedron have? 

{\bf Hint}: find various equations that relate the numbers $v$, $e$, $f$, $t$, $h$ of vertices, edges, faces, triangular faces, hexagonal faces, respectively.


%%%% DECIDE WHICH OF THE FOLLOWING COLORING PROBLEMS TO DO...

\item (20 points; 10 points each) Let $G$ be a graph. Recall that the chromatic polynomial $\chi(G,k)$ was defined to be the polynomial in $k$ which counts the number of proper vertex-colorings of $G$ with $k$ colors when $k$ is a positive integer.
\begin{enumerate}
\item Let $T$ be a tree on $n$ vertices. Compute $\chi(T,k)$. \\
{\bf Hint:} The answer only depends on $n$, not which particular tree on $n$ vertices $T$ is.
\item Let $C_n$ be the cycle graph on $n$ vertices. Compute $\chi(C_n,k)$.
\end{enumerate}

%\item (20 points) Let $n=2m$ be an even number, with $m > 1$. Let $G$ be the graph on $n$ vertices obtained from the complete graph $K_n$ by removing the edges of a Hamiltonian cycle. For example, in the cases $n=4$ and $n=6$, $G$ looks like the following:
%\[ \begin{tikzpicture}
%\node[draw, circle] (1) at (0:1) {1};
%\node[draw, circle] (2) at (90:1) {2};
%\node[draw, circle] (3) at (180:1) {3};
%\node[draw, circle] (4) at (270:1) {4};
%\node at (270:1.5) {$n=4$};
%\draw[thick] (1)--(3);
%\draw[thick] (2)--(4);
%\end{tikzpicture} \qquad \qquad \qquad \begin{tikzpicture}
%\node[draw, circle] (1) at (0:1) {1};
%\node[draw, circle] (2) at (60:1) {2};
%\node[draw, circle] (3) at (120:1) {3};
%\node[draw, circle] (4) at (180:1) {4};
%\node[draw, circle] (5) at (240:1) {5};
%\node[draw, circle] (6) at (300:1) {6};
%\node at (270:1.5) {$n=6$};
%\draw[thick] (1)--(3)--(5)--(1);
%\draw[thick] (2)--(4)--(6)--(2);
%\draw[thick] (1)--(4);
%\draw[thick] (2)--(5);
%\draw[thick] (3)--(6);
%\end{tikzpicture} \]
%What is the chromatic number of $G$?

\end{enumerate}


\end{document}
