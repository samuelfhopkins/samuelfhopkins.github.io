\documentclass[11pt]{article}

\usepackage{amsmath}
\usepackage{amssymb}

\usepackage{amsthm}

\usepackage{graphicx}
\usepackage{tikz}

\usepackage{ytableau}


\title{Ramsey theory, \\ UMTYMP Advanced Topics, Fall 2020}
\date{}


\begin{document}


\maketitle

\vspace{-0.5in}

\thispagestyle{empty}

In lecture today we discussed Ramsey's theorem, which says that if we color the edges of a big enough complete graph red and blue, then it will contain a large monochromatic complete subgraph. But we could imagine instead of coloring the edges, we color the $K_r$-subgraphs, and again look for a large complete subgraph all of whose $K_r$-subgraphs are colored the same color. This is the content of the \emph{generalized Ramsey's theorem}:

\medskip

\noindent {\bf Generalized (``hypergraph'') Ramsey's theorem.} For any $r\geq 1$ and $k,\ell\geq r$, there exists a smallest integer $R_r(k,\ell)$ such that if we color all the $K_r$-subgraphs of $K_N$ blue or red, where $N \geq R_r(k,\ell)$, then there exists either a $K_k$-subgraph of $K_N$ all of whose $K_r$-subgraphs are colored blue, or a $K_{\ell}$-subgraph of $K_N$ all of whose $K_r$-subgraphs are colored red.

\medskip

For example, the case $r=2$ is about coloring edges, so $R_2(k,\ell)=R(k,\ell)$; and the case $r=3$ is about coloring triangles. The proof is similar to the proof of Ramsey's theorem we gave, but with more notation. 

\medskip

Now we'll show an application of the case $r=3$ to geometry. Let $ES(n)$ denote the smallest integer such that if we choose any $N \geq ES(n)$ points in the plane $\mathbb{R}^2$, with no three points collinear, then there are $n$ of them which form the vertices of a {\bf convex} $n$-gon. (We will show $ES(n)$ exists!)

\begin{enumerate}
\item Show $ES(4)=5$ by drawing the possible point configurations.
\item Fix a configuration of points $1,2,\ldots,N$ in the plane, no three of them collinear, and for each $1 \leq i < j < k \leq N$, color the triangle $(i,j,k)$ blue if $i,j,k$ appear in clockwise order in the triangle they form, and red if they appear in counterclockwise order. Show that if you take any four points $1 \leq i < j < k < \ell \leq N$ such that all four triangles $(i,j,k)$, $(i,j,\ell)$, $(i,k,\ell)$, $(j,k,\ell)$ are the same color, then $i,j,k,\ell$ are the vertices of a {\bf convex} quadrilateral.
\item Conclude that $ES(n) \leq R_3(n,n)$.
\end{enumerate}

\medskip

This geometry problem is called the {\bf ``happy ending problem''}, because it lead to the marriage of George Szekeres and Esther Klein.

\end{document}
