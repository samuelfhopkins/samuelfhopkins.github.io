\documentclass[11pt]{article}

\usepackage{amsmath}
\usepackage{amssymb}


\title{Generating functions, \\UMTYMP Advanced Topics, Fall 2020}
\date{}

\begin{document}

\maketitle

\thispagestyle{empty}

\vspace{-0.5cm}

Let $F_0=0,F_1=1,F_2=1,F_3=2,F_4=3,...$ with $F_{n} = F_{n-1} + F_{n-2}$ be the Fibonacci number sequence. We saw that the generating function $F(x) := \sum_{n=0}^{\infty}F_0 x^n$ for the Fibonacci numbers is $F(x) = \frac{x}{1-x-x^2}$ and, via partial fraction decomposition, $F(x) = \frac{ 1/\sqrt{5}}{1- x(1+\sqrt{5})/2} - \frac{ 1/\sqrt{5}}{1- x(1-\sqrt{5})/2}$. Using the geometric series $\frac{1}{1-r}= \sum_{n=0}^{\infty}r^n$, we got $F_n = \frac{1}{\sqrt{5}}(\frac{1+\sqrt{5}}{2})^n - \frac{1}{\sqrt{5}}(\frac{1-\sqrt{5}}{2})^n$ and thus $F_n \approx \frac{1}{\sqrt{5}}\phi^n$ where $\phi =(\frac{1+\sqrt{5}}{2})\approx 1.618...$ is the golden ratio.

\begin{enumerate}
\item Consider the sequence $J_0=0, J_1=1, J_2=1, J_3=3, J_4=5, ...$ with $J_{n}=J_{n-1}+2J_{n-2}$. Form the generating function $J(x) := \sum_{n=0}^{\infty} J_n x^n$. Show that $J(x) = \frac{x}{1-x-2x^2}$.
\item Use partial fractions to show that $J(x) = \frac{1/3}{1-2x} - \frac{1/3}{1+x}$.
\item Conclude that $J_n = \frac{2^n-(-1)^n}{3}$, so that $J_n \approx \frac{1}{3}2^n$.
\end{enumerate}

For $n \in \mathbb{N}$, the generating function for the binomial coefficients $\binom{n}{k}$ is $\sum_{k=0}^{n} \binom{n}{k} x^k = (1+x)^n$ (by the binomial theorem). Generalizing this, for any real number $n \in \mathbb{R}$, define $\binom{n}{k} := \frac{n(n-1)(n-2)...(n-(k-1))}{k!}$. The \emph{generalized binomial theorem} says that for any $n \in \mathbb{R}$ we have $\sum_{k=0}^{\infty} \binom{n}{k} x^k = (1+x)^n$.



\begin{enumerate}
\setcounter{enumi}{3}
\item Show that $(1+x)^{-4} = \sum_{k=0}^{\infty} (-1)^k \binom{4+k-1}{k} x^k$ and thus that $\left(\frac{1}{1-x}\right)^4 = \sum_{k=0}^{\infty} \binom{4+k-1}{k} x^k$. 

\item Explain how the previous result relates to the ``giving pennies to kids''/``choosing bagels'' problem (hint: this case = 4 flavors of bagels).

\item Show that $(1+x)^{-1/2} = \sum_{k=0}^{\infty}(-1)^k \frac{1 \cdot 3 \cdot 5 \cdots (2k-1)}{2^k k!} x^k$ and thus that $\frac{1}{\sqrt{1-4x}} =  \sum_{k=0}^{\infty} \binom{2k}{k}x^k$ is the \emph{central binomial coefficient} generating function.

\item Use the previous result to show that for all $k \in \mathbb{N}$, $4^k = \sum_{j=0}^{k} \binom{2j}{j} \binom{2(k-j)}{k-j}$.

\end{enumerate}

\end{document}
