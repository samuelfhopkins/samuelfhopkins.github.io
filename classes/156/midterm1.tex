\documentclass[11pt]{article}
\usepackage[top=1in, bottom=1in, left=1in, right=1in]{geometry}

\usepackage{amsmath}
\usepackage{amssymb}
\usepackage{graphicx}

\title{Midterm \#1, 9/24 \\ Math 156 (Calculus I), Fall 2024}
\date{}

\begin{document}

\maketitle

\thispagestyle{empty}

\vspace{-1cm}

Each problem is worth 10 points, for a total of 50 points. You have 50 minutes to do the exam. Remember to \emph{show your work} on all problems! Where possible, simplify answers.

\begin{enumerate}
\item Let $f(x) = \cos(2x)-1$.
\begin{enumerate}
\item Graph $y=f(x)$. Be sure to include some value labels on your $x$- and $y$-axes.
\item Let $g(x)$ be the function whose graph is obtained from the graph of $f(x)$ by translating to the right by $\frac{\pi}{4}$ and stretching vertically by a factor of $3$. Write the formula for $g(x)$. (The formula you write should not have $f$ in it.)
\end{enumerate}
\item Let $g(x)=e^{5x} + 2$. 
\begin{enumerate}
\item Describe all the horizontal and/or vertical asymptotes of the graph $y=g(x)$ of this function. Explain your answer by saying what these asymptotes mean in terms of limits.
\item Let $f(x) = \ln(x-2)$. Write the formula for the composition $(f\circ g)(x)$. Make sure your formula is written in the most simplified form possible.
\end{enumerate} 
\item Let $f(x) = \displaystyle \frac{x^2-2x}{x^2-4}$. Compute the following limits, or if they do not exist explain why:
\begin{enumerate}
\item $\displaystyle \lim_{x \to 2} \, f(x)$
\item $\displaystyle \lim_{x \to 0} \, f(x)$
\item $\displaystyle \lim_{x \to -2} \, f(x)$
\end{enumerate}
\item Compute the following limits, or if they do not exist explain why:
\begin{enumerate}
\item $\displaystyle \lim_{x \to 0} \, e^{\sin(x)}$
\item $\displaystyle \lim_{x \to \infty} \, \frac{x^2-3x+2}{2x^2+2x-7}$
\item $\displaystyle \lim_{x \to \infty} \, \frac{x^2+2x-8}{7x+9}$
\end{enumerate}
\item What is the slope of the line tangent to the curve $y=-x^2+1$ at the point $(x,y)=(0,1)$? Explain your answer, for instance by sketching a graph or by discussing a limit.
\end{enumerate}

\end{document}