\documentclass[11pt]{article}
\usepackage[top=1in, bottom=1in, left=1in, right=1in]{geometry}

\usepackage{amsmath}
\usepackage{amssymb}
\usepackage{graphicx}

\title{Midterm \#1 Study Guide \\ Math 156 (Calculus I), Fall 2024}
\date{}

\begin{document}

\maketitle

\thispagestyle{empty}

\vspace{-1.5cm}

\begin{enumerate}
\item Basics (domain/range, what graph looks like, etc.)~for standard functions [\S1.1, 1.2, 1.4, 1.5]
\begin{enumerate}
\item algebraic functions: power functions (like $x^3$), root functions (like $\sqrt{x}$), \\ polynomials (like $x^2-3x+1$), rational functions (like $\frac{x^2-1}{x+5}$)
\item trigonometric functions (like $\sin(x)$ and $\cos(x)$)
\item exponential functions (like $e^x$) and logarithmic functions (like $\ln(x)$)
\item piecewise functions (like absolute value $|x|$)
\end{enumerate}
\item Algebraic operations on functions as geometric operations on graphs [\S1.3]
\begin{enumerate}
\item translation (up/down $\&$ left/right), stretching (horiz. $\&$ vert.), reflection (over axes)
\item symmetry under these operations, especially even and odd functions
\end{enumerate}
\item How to make new functions from old functions $f(x), g(x)$ [\S1.3]
\begin{enumerate}
\item sum ($f+g$), difference ($f-g$), scaling ($cf$), product ($fg$), quotient ($f/g$)
\item composition of functions: $(f \circ g)(x) = f(g(x))$
\end{enumerate}
\item Inverse functions $f=g^{-1}$ [\S1.5]
\begin{enumerate}
\item especially exponential and logarithmic functions
\item graph of inverse function is reflection across line $y=x$
\end{enumerate}
\item Intuitive definition of limit and basic reasons why a limit might not exist [\S2.2]
\begin{enumerate}
\item $\lim_{x\to a} f(x) = L$ means can make $f(x)$ arbitrarily close to $L$ by making $x \neq a$ close to $a$
\item one-sided limits $\lim_{x\to a^{\pm}} f(x)$: they must agree for usual (two-sided) limit to exist
\item $\lim_{x\to a} f(x) = \pm \infty$ counts as the limit not existing
\end{enumerate}
\item How to compute limits using the limit laws [\S2.3, 2.5]
\begin{enumerate}
\item sum ($f+g$), difference ($f-g$), scaling ($cf$), product ($fg$), quotient ($f/g$) limit laws
\item how to deal with ``$\frac{0}{0}$'' by cancelling factors
\item continuous functions (pushing limit thru, and direct substitution a.k.a.~``plugging in'')
\end{enumerate}
\item Limits at infinity and limits equal to infinity [\S2.2, 2.6]
\begin{enumerate}
\item limits at $\pm \infty$ = horizontal asymptotes (typical example: $\lim_{x\to -\infty} e^x = 0$)
\item $\pm \infty$-valued limits = vertical asymptotes (typical example: $\lim_{x\to 0^+} 1/x=\infty$)
\end{enumerate}
\item The definition(s) of derivative [\S2.1, 2.7, 2.8]
\begin{enumerate}
\item derivative as slope of the tangent to the curve $y=f(x)$ at a point $x=a$
\item derivative as a limit $f'(a) = \displaystyle \lim_{x \to a} \frac{f(x)-f(a)}{x-a}$
\end{enumerate}
\end{enumerate}

\end{document}