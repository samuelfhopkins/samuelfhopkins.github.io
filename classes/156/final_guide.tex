\documentclass[11pt]{article}
\usepackage[top=1in, bottom=1in, left=1in, right=1in]{geometry}

\usepackage{amsmath}
\usepackage{amssymb}
\usepackage{graphicx}

\title{Final Exam Study Guide \\ Math 156 (Calculus I), Fall 2022}
\date{}

\begin{document}

\maketitle

\pagestyle{empty}
\thispagestyle{empty}

\vspace{-2.3cm}

\begin{enumerate}
\item Basics (domain/range, what graph looks like, etc.)~for standard functions [\S1.1, 1.2, 1.4, 1.5]
\begin{enumerate}
\item algebraic functions: power functions (like $x^3$), root functions (like $\sqrt{x}$), \\ polynomials (like $x^2-3x+1$), rational functions (like $(x^2-1)/(x+5)$)
\item trigonometric functions (like $\sin(x)$ and $\cos(x)$)
\item exponential functions (like $e^x$) and logarithmic functions (like $\ln(x)$)
\item piecewise functions (like absolute value $|x|$)
\end{enumerate}

\item Algebraic operations on functions as geometric operations on graphs [\S1.3]
\begin{enumerate}
\item translation (up/down $\&$ left/right), stretching (horiz. $\&$ vert.), reflection (over axes)
\item symmetry under these operations, especially even and odd functions
\end{enumerate}

\item How to make new functions from old functions $f(x), g(x)$ [\S1.3]
\begin{enumerate}
\item sum ($f+g$), difference ($f-g$), scaling ($cf$), product ($fg$), quotient ($f/g$)
\item composition of functions: $(f \circ g)(x) = f(g(x))$
\end{enumerate}

\item Inverse functions $f=g^{-1}$ [\S1.5]
\begin{enumerate}
\item especially exponential and logarithmic functions
\item graph of inverse function is reflection across line $y=x$
\end{enumerate}

\item Intuitive definition of limit and basic reasons why a limit might not exist [\S2.2]
\begin{enumerate}
\item $\lim_{x\to a} f(x) = L$ means can make $f(x)$ arbitrarily close to $L$ by making $x \neq a$ close to $a$
\item one-sided limits $\lim_{x\to a^{\pm}} f(x)$: they must agree for usual (two-sided) limit to exist
\end{enumerate}

\item How to compute limits using the limit laws [\S2.3, 2.5]
\begin{enumerate}
\item sum ($f+g$), difference ($f-g$), scaling ($cf$), product ($fg$), quotient ($f/g$) limit laws
\item how to deal with ``$0/0$'' by cancelling factors
\item continuous functions (pushing limit thru, and direct substitution a.k.a.~``plugging in'')
\end{enumerate}

\item Limits at infinity and limits equal to infinity [\S2.2, 2.6]
\begin{enumerate}
\item limits at $\pm \infty$ = horizontal asymptotes
\item $\pm \infty$-valued limits = vertical asymptotes
\end{enumerate}

\item The definition(s) of derivative [\S2.1, 2.7, 2.8]
\begin{enumerate}
\item derivative as slope of the tangent to a curve at a point
\item derivative as a limit $f'(a) = \lim_{x \to a} \; (f(x)-f(a))/(x-a)$
\end{enumerate}

\item Derivatives of basic functions [\S 3.1, 3.3, 3.6]
\begin{enumerate}
\item power functions: $d/dx (x^n) = n x^{n-1}$
\item exponential and logarithmic functions: $d/dx(e^x) = e^x$ and $d/dx(\ln(x)) = 1/x$
\item trigonometric functions: $d/dx(\sin(x)) = \cos(x)$ and $d/dx(\cos(x)) = -\sin(x)$
\end{enumerate}

\item Rules for derivatives of combinations of functions [\S 3.1, 3.2, 3.4]
\begin{enumerate}
\item derivative is linear: $d/dx( \; a\cdot f(x) + b\cdot g(x) \;) = a \cdot f'(x) + b \cdot g'(x)$ for $a,b \in \mathbb{R}$
\item product rule: $d/dx( \; f(x)\cdot g(x) \;) = f(x) \cdot g'(x) + g(x) \cdot f'(x)$
\item chain rule: $d/dx (\; f(g(x)) \; ) = f'(g(x)) \cdot g'(x)$
\item quotient rule: $\displaystyle d/dx (\; f(x)/g(x) \;) = (\; g(x)\cdot f'(x) - f(x)\cdot g'(x) \; )/(g(x))^2$\\ {[\emph{don't have to separately memorize quotient rule, it follows from other rules}]}
\end{enumerate}

\item Implicit differentiation and related rates [\S 3.5, 3.9]
\begin{enumerate}
\item for $y$ defined implicitly via equation $p(x,y)=0$, find $dy/dx$ by taking $d/dx$ of both sides, and use this to find the slope of the tangent at any point on the curve
\item if two quantities $f(t), g(t)$ are related, then their rates of change $df/dt$, $dg/dt$ are related: like with implicit differentiation, just differentiate the relation between $f(t)$ and $g(t)$
\end{enumerate}

\item Linear approximation [\S 3.10]
\begin{enumerate}
\item tangent is best linear approximation to $f(x)$ near a point $a$: $f(x) \approx f(a) + (x-a) \cdot f'(a)$
\end{enumerate}

\item Extreme values [\S 4.1, 4.3]
\begin{enumerate}
\item local versus absolute (global) minimum and maximum values, Extreme Value Theorem
\item the Closed Interval Method: extreme values of continuous $f$ on closed interval must occur at endpoints or at critical points (values $x$ where $f'(x)=0$ or is not defined)
\item 1st and 2nd Derivative Tests for deciding if critical points are min.'s or max.'s
\end{enumerate}

\item What derivatives tell us about shape of graph [\S 4.2, 4.3, 4.5]
\begin{enumerate}
\item $f'(x)>0$ means $f$ is increasing, $f'(x) <0$ means $f$ is decreasing
\item $f''(x)>0$ means $f$ is concave up (smile), $f''(x) <0$ means $f$ is concave down (frown)
\end{enumerate}

\item L'H\^{o}pital's rule [\S 4.4]
\begin{enumerate}
\item for indeterminate form limits (meaning ``$\pm \frac{\infty}{\infty}$'' or ``$\frac{0}{0}$''), $\lim_{x\to a} \frac{f(x)}{g(x)} = \lim_{x\to a} \frac{f'(x)}{g'(x)}$
\end{enumerate}

\item Anti-derivatives, a.k.a.~indefinite integrals [\S 4.9, 5.4, 5.5]
\begin{enumerate}
\item basic anti-derivatives/indefinite integrals: $\int x^n \; dx = \frac{1}{n+1}x^{n+1}+C$, $\int e^x \; dx = e^x + C$, $\int \frac{1}{x} \; dx = \ln(x) + C$, $\int \sin(x) \; dx = -\cos(x) +C$, $\int \cos(x) \; dx = \sin(x) + C$
\item integral is linear: $\int a\cdot f(x) + b\cdot g(x) \; dx = a\int f(x) \; dx + b \int g(x) \; dx$ for $a,b\in \mathbb{R}$
\item the $u$-substitution technique: can treat the ``$dx$'' in an integral as a differential
\end{enumerate}

\item Definite integrals [\S 5.1, 5.2, 5.3]
\begin{enumerate}
\item definite integral $\int_{a}^{b} f(x) \; dx$ as area under the curve $y=f(x)$ from $x=a$ to $x=b$, or more precisely as limit of ``Riemann'' (rectangle) sums $\lim_{n \to \infty} \sum_{i=0}^{n} f(x^{*}_i) \Delta x$
\item Fundamental Theorem of Calculus: $\int_{a}^{b} f(x) \; dx = F(b)-F(a) =\int f(x) \; dx \big ]_{a}^{b}$ 
\end{enumerate}

\end{enumerate}

\end{document}