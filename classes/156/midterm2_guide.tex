\documentclass[11pt]{article}
\usepackage[top=1in, bottom=1in, left=1in, right=1in]{geometry}

\usepackage{amsmath}
\usepackage{amssymb}
\usepackage{graphicx}

\title{Midterm \#2 Study Guide \\ Math 156 (Calculus I), Fall 2022}
\date{}

\begin{document}

\maketitle

\thispagestyle{empty}

\vspace{-1.75cm}

\begin{enumerate}
\item Derivatives of basic functions [\S 3.1, 3.3, 3.6]
\begin{enumerate}
\item power functions: $d/dx (x^n) = n x^{n-1}$
\item exponential and logarithmic functions: $d/dx(e^x) = e^x$ and $d/dx(\ln(x)) = 1/x$
\item trigonometric functions: $d/dx(\sin(x)) = \cos(x)$ and $d/dx(\cos(x)) = -\sin(x)$
\end{enumerate}

\item Rules for derivatives of combinations of functions [\S 3.1, 3.2, 3.4]
\begin{enumerate}
\item derivative is linear: $d/dx( \; a\cdot f(x) + b\cdot g(x) \;) = a \cdot f'(x) + b \cdot g'(x)$ for $a,b \in \mathbb{R}$
\item product rule: $d/dx( \; f(x)\cdot g(x) \;) = f(x) \cdot g'(x) + g(x) \cdot f'(x)$
\item chain rule: $d/dx (\; f(g(x)) \; ) = f'(g(x)) \cdot g'(x)$
\item quotient rule: $\displaystyle d/dx (\; f(x)/g(x) \;) = \frac{g(x)\cdot f'(x) - f(x)\cdot g'(x)}{(g(x))^2}$\\ {[\emph{don't have to separately memorize quotient rule, it follows from other rules}]}
\end{enumerate}

\item Implicit differentiation and related rates [\S 3.5, 3.9]
\begin{enumerate}
\item for $y$ defined implicitly via equation $p(x,y)=0$, find $dy/dx$ by taking $d/dx$ of both sides, and use this to find the slope of the tangent at any point on the curve
\item if two quantities $f(t), g(t)$ are related, then their rates of change $df/dt$, $dg/dt$ are related: like with implicit differentiation, just differentiate the relation between $f(t)$ and $g(t)$
\end{enumerate}

\item Linear approximation [\S 3.10]
\begin{enumerate}
\item tangent is best linear approximation to $f(x)$ near a point $a$: $f(x) \approx f(a) + (x-a) \cdot f'(a)$
\end{enumerate}

\item Extreme values [\S 4.1, 4.3]
\begin{enumerate}
\item local versus absolute (global) minimum and maximum values, Extreme Value Theorem
\item the Closed Interval Method: extreme values of continuous $f$ on closed interval must occur at endpoints or at critical points (values $x$ where $f'(x)=0$ or is not defined)
\item 1st and 2nd Derivative Tests for deciding if critical points are min.'s or max.'s
\end{enumerate}

\item What derivatives tell us about shape of graph [\S 4.2, 4.3, 4.5]
\begin{enumerate}
\item $f'(x)>0$ means $f$ is increasing, $f'(x) <0$ means $f$ is decreasing
\item $f''(x)>0$ means $f$ is concave up (smile), $f''(x) <0$ means $f$ is concave down (frown)
\end{enumerate}

\item L'H\^{o}pital's rule [\S 4.4]
\begin{enumerate}
\item for indeterminate form limits (meaning ``$\pm \frac{\infty}{\infty}$'' or ``$\frac{0}{0}$''), $\displaystyle \lim_{x\to a} \frac{f(x)}{g(x)} = \lim_{x\to a} \frac{f'(x)}{g'(x)}$
\end{enumerate}

\end{enumerate}

\end{document}