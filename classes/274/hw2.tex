\documentclass[11pt]{article}
\usepackage[top=1in, bottom=1in, left=1in, right=1in]{geometry}

\usepackage{amsmath}
\usepackage{amssymb}

\usepackage{hyperref}

\title{Howard Math 274, HW\# 2, \\ {\normalsize Spring 2022; Instructor: Sam Hopkins; Due: Friday, March 25th}}
\date{}

\begin{document}

\maketitle

\thispagestyle{empty}

\vspace{-1.5cm}

\begin{enumerate}

\item Let $\lambda=(\lambda_1,\lambda_2,\ldots), \mu=(\mu_1,\mu_2,\ldots) \vdash n$ be partitions of $n$. Recall that the \emph{lexicographic order} $\prec$ on partitions of $n$ is given by $\mu \prec \lambda$ iff there is some $j$ such that $\mu_i=\lambda_i$ for all $i < j$ and $\mu_j < \lambda_j$. It is a total order: we either have $\mu \prec \lambda$ or $\lambda \prec \mu$ or $\lambda = \mu$.

A different order on partitions of $n$ is the dominance order. The \emph{dominance order} $\leq$ is defined by $\mu \leq \lambda$ iff $\mu_1 + \mu_2 + \cdots + \mu_j \leq \lambda_1 + \lambda_2 + \cdots + \lambda_j$ for all $j$. The dominance order is only partial order: we might have neither $\mu \leq \lambda$ nor $\lambda \leq \mu$.

\begin{enumerate}
\item Show that the lexicographic order \emph{extends} the dominance order in the sense that if we have partitions $\lambda, \mu \vdash n$ with $\mu \leq \lambda$ and $\mu \neq \lambda$ then necessarily $\mu \prec \lambda$.
\item Give an example of partitions $\lambda,\mu \vdash n$ with $\mu \prec \lambda$ but $\mu \not \leq \lambda$.
\end{enumerate}

\item Show that we could've used dominance order instead of lexicographic order in our arguments about the triangularity of the transition matrices from $p_{\lambda}$ or $e_{\lambda}$ to $m_{\mu}$. That is, show that
\[ p_{\lambda} = \sum_{\lambda \leq \mu} \alpha^{\lambda}_{\mu} \; m_{\mu} \qquad \textrm{and} \qquad e_{\lambda} = \sum_{\mu \leq \lambda^t} \beta^{\lambda}_{\mu} \; m_{\mu} \quad \textrm{for coefficients $\alpha^{\lambda}_{\mu},\beta^{\lambda}_{\mu}\in\mathbb{C}$}\]
for any $\lambda \vdash n$, where $\leq$ is dominance order and $\lambda^t$ is the transpose (a.k.a.~conjugate) of $\lambda$.

\item Let $\lambda \vdash n$ and define $f^{\lambda}$ to be the coefficient of $x_1x_2\cdots x_n$ in the Schur function $s_{\lambda}(x_1,x_2,\ldots)$. Explain why $f^{\lambda} = f^{\lambda^t}$. Give an example showing that this is not true for other coefficients of Schur functions (i.e., that $s_{\lambda} \neq s_{\lambda^t}$ in general).

\item The Cauchy--Binet formula says that if $A=(A_{i,j})$ is an $m \times n$ matrix and $B=(B_{i,j})$ is an $n\times m$ matrix, then the determinant of the $m \times m$ matrix $AB$ can be computed by
\[ \det(AB) = \sum_{I \subseteq [n], \; \#I = m} \det(A\mid_{\mathrm{cols}=I})  \det(B\mid_{\mathrm{rows}=I}). \]
Here, as always, $[n] := \{1,2,\ldots,n\}$, and $A\mid_{\mathrm{cols}=I}$ (resp., $B\mid_{\mathrm{rows}=I}$) means the $m\times m$ matrix we get by restricting $A$ to the columns in $I$ (resp., by restricting $B$ to the rows in $I$).

Deduce the Cauchy--Binet formula from the Lindstr\"{o}m--Gessel--Viennot formula.

{\bf Hint}: Consider the network with source vertices $s_1,\ldots,s_m$, target vertices $t_1,\ldots,t_m$, and internal vertices $k_1,\ldots, k_n$, and edges $s_i \to k_j$ with weight $A_{i,j}$ and $k_i \to t_j$ with weight $B_{i,j}$.

\item Let $\lambda=(\lambda_1,\lambda_2,\ldots)$ be a partition and $k$ a positive integer. Give a formula for $m_{\lambda}(\overbrace{1,1,\ldots,1}^{\textrm{$k$ $1$'s}})$.

{\bf Hint}: Your formula can use the \emph{length} $\ell(\lambda) := \max \{ i\colon \lambda_i > 0 \}$ of the partition $\lambda$, as well as the \emph{multiplicities} $m_i(\lambda) := \{j\colon \lambda_j = i\}$ for $i \geq 1$.

\end{enumerate}


\end{document}