\documentclass[11pt]{article}
\usepackage[top=1in, bottom=1in, left=1in, right=1in]{geometry}

\usepackage{amsmath}
\usepackage{amssymb}

\title{Howard Math 273, HW\# 2, \\ {\normalsize Fall 2021; Instructor: Sam Hopkins; Due: Friday, November 5th}}
\date{}

\begin{document}

\maketitle

\thispagestyle{empty}

\vspace{-1.8cm}

\begin{enumerate}

\item Fix a positive integer $k$. We showed the ordinary generating function $F_k(x) := \sum_{n\geq 0}S(n,k)x^n$ of the Stirling numbers of the 2nd kind satisfies $F_k(x) = \frac{x^k}{(1-x)(1-2x)\cdots(1-kx)}$. Find the partial fraction decomposition of $F_k(x)$, i..e., the coefficients $a_j \in \mathbb{R}$, $j=1,2,\ldots,k$, for which $F_k(x) = \frac{a_1}{(1-x)} + \frac{a_2}{(1-2x)} + \cdots + \frac{a_k}{(1-kx)}$. Conclude $S(n,k) = \sum_{j=0}^{k} a_j \cdot j^n$. \\
{\bf Hint}: clear denominators, and then plug in $x=\frac{1}{1}, \frac{1}{2}, \frac{1}{3}, \cdots, \frac{1}{k}$. \\
{\bf Bonus just to think about, not do}: prove $S(n,k) = \sum_{j=0}^{k} a_j \cdot j^n$ using (i) the \emph{exponential} g.f.~$\sum_{n \geq 0} S(n,k) \frac{x^n}{n!} = \frac{1}{k!}(e^x-1)^k$; or (ii) the Principle of Inclusion-Exclusion (P.I.E.).

\item \emph{(Stanley, EC1, \#2.2)} Let $A$ be some finite set of objects, and suppose these objects potentially posses $n$ different \emph{properties} $p_1,p_2,\ldots,p_n$: e.g., $p_1=\textrm{``is green''}$; $p_2=\textrm{``is solid''}$; et cetera.  For~$X\subseteq [n]$, let $f_{=}(X)$ denote the number of elements in $A$ possessing \emph{exactly} the properties $p_i$ for $i\in X$ (and not possessing any of the properties $p_j$ for $j\notin X$); and let $f_{\geq}(X)$ denote the number of elements in $A$ possessing \emph{at least} the properties $p_i$ for $i\in X$ (but potentially also some properties $p_j$ for $j\notin X$). Give a bijective proof of the P.I.E.~identity
\[ \sum_{X\subseteq [n]} f_{=}(X) (1+y)^{\# X} = \sum_{Y\subseteq [n]} f_{\geq}(Y) y^{\#Y},\]
i.e., give a bijective proof, for each $k$, that the coefficients of $y^k$ on the L- and RHS are equal.

\item \emph{(Stanley, EC1, \#2.25(a))} Let $f_i(m,n)$ be the number of $m\times n$ matrices of $0$'s and $1$'s, with a total of $i$ $1$'s, and with at least one $1$ in each row and column. Use the P.I.E.~to show
\[ \sum_{i \geq 0} f_i(m,n) t^i = \sum_{k=0}^{n}(-1)^k\binom{n}{k}( (1+t)^{n-k}-1)^m.\]

\item \emph{(Stanley, EC1, \#2.25(b))} With $f_i(m,n)$ as in the previous problem, show that
\[ \sum_{m,n \geq 0} \left(\sum_{i \geq 0} f_i(m,n) t^i \right) \frac{x^m \, y^n} {m! \, n!} = e^{-x-y} \cdot \sum_{m,n \geq 0}  (1+t)^{mn}  \frac{x^m \, y^n} {m! \, n!} .\]
{\bf Hint}: use the formula from the previous problem, and do some algebraic manipulations.


\renewcommand{\binom}{\genfrac{(}{)}{0pt}{}}
\DeclareRobustCommand{\qbinom}{\genfrac{\lbrack}{\rbrack}{0pt}{}}

\item The $q$-binomial coefficient satisfies $\qbinom{n}{k}_q = \sum_{w \in \mathcal{W}_{n,k}} q^{\mathrm{inv}(w)},$ where $\mathcal{W}_{n,k}$ is the set of words that are rearrangements of $(n-k)$ $0$'s, and $k$ $1$'s, and $\mathrm{inv}(w)$ is the number of inversions of~$w$.

{\bf Suppose $n=2m$ is even.} Prove that $\qbinom{n}{k}_{q := -1}$ (the evaluation of the $q$-binomial at $q=-1$) is equal to $\#\mathcal{P}_{n,k}$, where $\mathcal{P}_{n,k}$ is the subset of words $w=w_1w_2\ldots w_{n} \in \mathcal{W}_{n,k}$ that are \emph{palindromes} (i.e., which satisfy $w_i = w_{n+1-i}$ for all $i$). Do this by defining a {\bf sign-reversing involution}. That is, define an involution $\tau\colon \mathcal{W}_{n,k}\to \mathcal{W}_{n,k}$ satisfying:
\begin{itemize}
\item $\mathrm{inv}(w)$ and $\mathrm{inv}(\tau(w))$ have opposite parity for all $w\in \mathcal{W}_{n,k}$ with $\tau(w)\neq w$;
\item $\mathrm{inv}(w)$ is even for all $w\in \mathcal{W}_{n,k}$ with $\tau(w)=w$;
\item $\#\{w\in \mathcal{W}_{n,k}\colon \tau(w)=w\} = \#\mathcal{P}_{n,k}$.
\end{itemize}

\end{enumerate}


\end{document}
