\documentclass[11pt]{article}
\usepackage[top=1in, bottom=1in, left=1in, right=1in]{geometry}

\usepackage{amsmath}
\usepackage{amssymb}

\title{Howard Math 273, HW\# 2, \\ {\normalsize Fall 2023; Instructor: Sam Hopkins; Due: Friday, November 3rd}}
\date{}

\begin{document}

\maketitle

\thispagestyle{empty}

\vspace{-1.8cm}

\begin{enumerate}

\item Recall the Stirling numbers of the $2$nd kind are $S(n,k) := \#$ set partitions of $\{1,2,\ldots,n\}$ into $k$ (non-empty) blocks. We saw $F_k(x) := \sum_{n\geq 0}S(n,k)x^n$ satisfies $F_k(x) = \frac{x^k}{(1-x)(1-2x)\cdots(1-kx)}$ for any $k \geq 1$. Find the partial fraction decomposition of $F_k(x)$, i.e., find the numbers $a_j \in \mathbb{Q}$ for which $F_k(x) = a_0 + \frac{a_1}{(1-x)} + \frac{a_2}{(1-2x)} + \cdots + \frac{a_k}{(1-kx)}$. Conclude that $S(n,k) = \sum_{j=0}^{k} a_j \cdot j^n$. \\
{\bf Hint}: Clear denominators, and then plug in $x=\frac{1}{1}, \frac{1}{2}, \frac{1}{3}, \cdots, \frac{1}{k}$ and finally $x=0$.

\item \emph{(Stanley, EC1, \#2.9)} Another way to find the $a_j$ from the previous problem is using the Principle of Inclusion-Exclusion (P.I.E.), as this problem will show. Let $\widehat{S}(n,k) := k! \cdot S(n,k)$. 
\begin{enumerate}
\item Explain why $\widehat{S}(n,k)$ is the number of ways to place $n$ labelled balls into $k$ labelled boxes so that all boxes are non-empty.
\item How many ways are there to place $n$ labelled balls into $k$ labelled boxes so that the boxes labeled $i_1,i_2,\ldots,i_j$ are empty (but the other boxes may be empty or not)?
\item Use parts (a), (b), and the P.I.E. to conclude that $\widehat{S}(n,k) = \sum_{j=0}^{k} (-1)^j \binom{k}{j}(k-j)^n$.
\end{enumerate}

\item \emph{(Stanley, EC1, \#2.25(a))} Let $f_i(m,n)$ be the number of $m\times n$ matrices of $0$'s and $1$'s, with a total of $i$ $1$'s, and with at least one $1$ in each row and column. Use the P.I.E.~to show that
\[ \sum_{i \geq 0} f_i(m,n) t^i = \sum_{k=0}^{n}(-1)^k\binom{n}{k}( (1+t)^{n-k}-1)^m.\]

\item \emph{(Stanley, EC1, \#2.25(b))} With $f_i(m,n)$ as in the previous problem, show that
\[ \sum_{m,n \geq 0} \left(\sum_{i \geq 0} f_i(m,n) t^i \right) \frac{x^m}{m!} \frac{y^n}{n!} = e^{-x-y} \cdot \sum_{m,n \geq 0}  (1+t)^{mn} \frac{x^m}{m!} \frac{y^n}{n!} .\]
{\bf Hint}: You can start with the formula from the previous problem, and then do some algebraic manipulations. Alternatively, you can use the theory of exponential generating functions.

\renewcommand{\binom}{\genfrac{(}{)}{0pt}{}}
\DeclareRobustCommand{\qbinom}{\genfrac{\lbrack}{\rbrack}{0pt}{}}

\item The $q$-binomial coefficient satisfies $\qbinom{n}{k}_q = \sum_{w \in \mathcal{W}_{n,k}} q^{\mathrm{inv}(w)},$ where $\mathcal{W}_{n,k}$ is the set of words that are rearrangements of $(n-k)$ $0$'s and $k$ $1$'s, and $\mathrm{inv}(w)$ is the number of inversions of~$w$.

{\bf Suppose $n=2m$ is even.} Prove that $\qbinom{n}{k}_{q := -1}$ (the evaluation of the $q$-binomial at $q=-1$) is equal to $\#\mathcal{P}_{n,k}$, where $\mathcal{P}_{n,k}$ is the subset of words $w=w_1w_2\ldots w_{n} \in \mathcal{W}_{n,k}$ that are \emph{palindromes}, i.e., which satisfy $w_i = w_{n+1-i}$ for all $i$. Do this by defining a {\bf sign-reversing involution}. That is, define an involution $\tau\colon \mathcal{W}_{n,k}\to \mathcal{W}_{n,k}$ satisfying:
\begin{itemize}
\item $\mathrm{inv}(w)$ and $\mathrm{inv}(\tau(w))$ have opposite parity for all $w\in \mathcal{W}_{n,k}$ with $\tau(w)\neq w$;
\item $\mathrm{inv}(w)$ is even for all $w\in \mathcal{W}_{n,k}$ with $\tau(w)=w$;
\item $\#\{w\in \mathcal{W}_{n,k}\colon \tau(w)=w\} = \#\mathcal{P}_{n,k}$.
\end{itemize}

\end{enumerate}


\end{document}
