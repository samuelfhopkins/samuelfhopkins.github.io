\documentclass[11pt]{article}
\usepackage[top=1in, bottom=1in, left=1in, right=1in]{geometry}

\usepackage{amsmath}
\usepackage{amssymb}

\title{Math 210 (Modern Algebra I), HW\# 4, \\ {\normalsize Fall 2025; Instructor: Sam Hopkins; Due: Wednesday, October 15th}}
\date{}

\begin{document}

\maketitle

\thispagestyle{empty}

Recall that all rings $R$ are assumed to be unital (i.e., have a $1$) but not necessarily commutative.

\begin{enumerate}

\item A ring $R$ which satisfies $a^2 = a$ for all $a\in R$ is called a \emph{Boolean ring}. 
\begin{enumerate}
\item Prove that a Boolean ring $R$ is commutative, and satisfies $a+a=0$ for all $a\in R$.
\item Let $U$ be a set and let $\mathcal{P}(U)$ denote the set of all subsets of $U$. For $A,B \in \mathcal{P}(U)$, define $A+B = (A\setminus B) \cup (B\setminus A)$ and $AB = A\cap B$. Prove that this gives $\mathcal{P}(U)$ the structure of a Boolean ring.
\end{enumerate}

\item Let $G$ be a finite group and $R = \mathbb{Q}[G]$ be the group algebra of $G$ over the rational numbers~$\mathbb{Q}$. Consider the element $x = \displaystyle \frac{1}{|G|}\sum_{g \in G} g \in R$. Prove that $x$ is an \emph{idempotent}, i.e., that $x^2 = x$.

\item Recall that the \emph{center} of a (noncommutative) ring $R$ is $Z(R) = \{x \in R\colon xy = yx \textrm{ for all $y \in R$}\}$. \\
\\ Now let $R$ be a commutative ring and consider the ring $M_n(R)$ of $n\times n$ matrices with entries in $R$. What is the center $Z(M_n(R))$ of this matrix ring?

\item Let $\mathbb{H}$ denote the quaternions. Recall that an element $p\in \mathbb{H}$ can be represented as a formal sum $p=a + b\mathbf{i} + c\mathbf{j} + d\mathbf{k}$, with $a,b,c,d\in \mathbb{R}$ real numbers, and with $\mathbf{i}^2=\mathbf{j}^2=\mathbf{k}^2=\mathbf{ijk}=-1$. Define the \emph{norm} of such an element $p$ to be $|p| = \sqrt{a^2+b^2+c^2+d^2}$. Prove that this norm is mulpticative, i.e., that $| p q | = |p| \, |q|$ for $p,q \in \mathbb{H}$.

\item Let $R$ be a commutative ring.
\begin{enumerate}
\item Let $I$ be an ideal of $R$ and define its \emph{radical} to be $\mathrm{Rad}(I) = \{x \in R\colon x^n \in I \textrm{ for some $n \geq 1$}\}$. Prove that $\mathrm{Rad}(I)$ is also an ideal of $R$. {\bf Hint}: feel free to use the binomial theorem.
\item Recall that $x\in R$ is \emph{nilpotent} if there is some $n \geq 1$ such that $x^n=0$. Prove that the collection of all nilpotent elements is an ideal of $R$. {\bf Hint}: you can use the previous part.
\end{enumerate}



\end{enumerate}


\end{document}
