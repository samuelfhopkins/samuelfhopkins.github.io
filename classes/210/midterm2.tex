\documentclass[11pt]{article}
\usepackage[top=1in, bottom=1in, left=1in, right=1in]{geometry}

\usepackage{amsmath}
\usepackage{amssymb}

\title{Math 210 (Modern Algebra I), Midterm \# 2, \\ {\normalsize Fall 2025; Instructor: Sam Hopkins; Taken on: Wednesday, November 19th}}
\date{}

\begin{document}

\maketitle

Each problem is worth 10 points, for a total of 50 points. You have 80 minutes to do the exam. Partial credit will be given generously, so write as much as you know for each problem.

\smallskip

Throughout we use $\mathbb{Z}$ for the integers, $\mathbb{Q}$ for the rationals, $\mathbb{R}$ for the real numbers, and $\mathbb{C}$ for the complex numbers.

\thispagestyle{empty}
\begin{enumerate}

\item Give a specific example of a noncommutative ring $R$ (with a $1$). Show that your example really is noncommutative by exhibiting two elements $x,y \in R$ with $xy \neq yx$.

\item Recall that the ring of \emph{Gaussian integers} is $\mathbb{Z}[i] = \{a+bi\colon a,b\in \mathbb{Z}\}$ where $i=\sqrt{-1} \in \mathbb{C}$. Consider the ideal $I = \langle 2, \, 3+i \rangle \subseteq \mathbb{Z}[i]$. We proved that $\mathbb{Z}[i]$ is a principal ideal domain. Hence there exists an $r \in \mathbb{Z}[i]$ for which $I = \langle r \rangle$. Find such an $r$. \\{\bf Hint}: how can the identity $(1-i)(1+i)=2$ help you?

\item
\begin{enumerate}
\item State what it means for an element $r \in R$ of a commutative ring $R$ to be irreducible.
\item State what it means for an element $r \in R$ of a commutative ring $R$ to be prime.
\item Is $x^2-1$ a prime element of the polynomial ring $\mathbb{Q}[x]$? Justify your answer.
\item Is $x^2+1$ a prime element of the polynomial ring $\mathbb{R}[x]$? Justify your answer.
\end{enumerate}

\item Consider the following sequence of abelian groups and homomorphisms between them:
\[ 0 \to \mathbb{Z} \xrightarrow{f} \mathbb{Z} \oplus \mathbb{Z} \xrightarrow{g} \mathbb{Z} \to 0.\]
Here $f\colon \mathbb{Z} \to \mathbb{Z} \oplus \mathbb{Z}$ is given by $f(a) = (a,a)$, and $g\colon \mathbb{Z} \oplus \mathbb{Z} \to \mathbb{Z}$ is given by $g(a,b) = a-b$. Prove that this is a short exact sequence.

\item Recall for a commutative ring $R$ and an $R$-module $M$, its \emph{dual module} is $M^* = \mathrm{Hom}_R(M,R)$. Give a specific example of a commutative ring $R$ and an $R$-module $M$ such that $M$ is not isomorphic to its double dual $M^{**}$. Explain why your example works.

\end{enumerate}


\end{document}
