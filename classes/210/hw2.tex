\documentclass[11pt]{article}
\usepackage[top=1in, bottom=1in, left=1in, right=1in]{geometry}

\usepackage{amsmath}
\usepackage{amssymb}

\title{Math 210 (Modern Algebra I), HW\# 2, \\ {\normalsize Fall 2024; Instructor: Sam Hopkins; Due: Wednesday, September 11th}}
\date{}

\begin{document}

\maketitle

\thispagestyle{empty}
\begin{enumerate}

\item Prove that the nontrivial groups with no proper, nontrivial subgroups are $\mathbb{Z}/p\mathbb{Z}$ for $p$~prime.

\item For a positive integer $n$, the \emph{multiplicative group} $(\mathbb{Z}/n\mathbb{Z})^\times$ consists of those $a\in \mathbb{Z}/n\mathbb{Z}$ satisfying $\gcd(a,n)=1$ (i.e., coprime to $n$), with product given by multiplication modulo~$n$. This group is \emph{not} the same as the additive group $\mathbb{Z}/n\mathbb{Z}$: e.g., the identity element in $(\mathbb{Z}/n\mathbb{Z})^\times$ is $1$.

Now let $p$ be a prime. Use Lagrange's Theorem for the group $(\mathbb{Z}/p\mathbb{Z})^\times$ to prove \emph{Fermat's Little Theorem}, which states that $a^p \equiv a \mod p$ for all $a\in \mathbb{Z}$.

\item \begin{enumerate}
\item Let $G$ be a (not necessarily finite!) group and $H \leq G$ a subgroup of $G$ with $[G:H]=2$. Prove that $H$ is a normal subgroup of $G$.

\item Give an example of a group $G$ and a subgroup $H \leq G$ with $[G:H]=3$ such that $H$ is not a normal subgroup of $G$.
\end{enumerate}

\item Let $D_n$ denote the dihedral group of symmetries of a regular $n$-gon. Prove that the map $\varphi\colon D_n \to \mathbb{Z}/2\mathbb{Z}$ which sends all reflections to $1$ and all other elements to~$0$ is a homomorphism. Explain why the kernel of $\varphi$ is isomorphic to $\mathbb{Z}/n\mathbb{Z}$.

\item Again letting $D_n$ denote the dihedral group, recall that in class we showed that $D_n$ has a presentation $D_n = \langle r,s \colon r^n = s^2 = (sr)^2=1\rangle$, where $r$ corresponds to clockwise rotation by $\frac{2\pi}{n}$ radians and $s$ corresponds to one of the reflections. Explain why we also have the presentation $D_n = \langle s,t \colon s^2 = t^2 = (st)^n = 1\rangle$.



\end{enumerate}


\end{document}
