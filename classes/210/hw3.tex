\documentclass[11pt]{article}
\usepackage[top=1in, bottom=1in, left=1in, right=1in]{geometry}

\usepackage{amsmath}
\usepackage{amssymb}

\title{Math 210 (Modern Algebra I), HW\# 3, \\ {\normalsize Fall 2024; Instructor: Sam Hopkins; Due: Wednesday, September 25th}}
\date{}

\begin{document}

\maketitle

\thispagestyle{empty}
\begin{enumerate}

\item For $p$ a prime number, a group $G$ is called a \emph{$p$-group} if every element has order a power of $p$. Prove that a finite abelian $p$-group is generated by its elements of maximal order.

\item Let $G$ be a group. Recall that an automorphism $\varphi\in \mathrm{Aut}(G)$ is called \emph{inner} if it is conjugation by some fixed $h\in G$, i.e., is of the form $\varphi\colon g\mapsto hgh^{-1}$. Also recall that the \emph{center} of $G$ is $Z(G) = \{g \in G\colon gx = xg \textrm{ for all $x\in G$}\}$. 
\begin{enumerate}
\item Prove that $\mathrm{Inn}(G)$, the set of inner automorphisms of $G$, is a subgroup of $\mathrm{Aut}(G)$. \\ (In fact it is a normal subgroup, but you do not need to prove that.)
\item Prove that $Z(G)$ is a normal subgroup of $G$.
\item Prove that $G/Z(G)$ is isomorphic to $\mathrm{Inn}(G)$.
\end{enumerate}

\item An action of a group $G$ on a set $S$ is called \emph{transitive} if for every $x,y \in S$ there is a $g\in G$ such that $g \cdot x = y$. An action of a group $G$ on a set $S$ is called \emph{free} if $g \cdot x = x$ for some $x\in S$ and $g\in G$ implies $g=e$. In what follows, let $S=\{1,2,\ldots,n\}$ and let $G$ be a finite group.
\begin{enumerate}
\item Suppose $G$ acts transitively on $S$. Prove that $n$ divides the order of $G$.
\item Suppose $G$ acts freely and transitively on $S$. Prove that the order of $G$ is exactly $n$.
\item Give an example, for each $n \geq 1$, of such a $G$ acting freely and transitively on $S$.
\end{enumerate}

\item The \emph{cycle type} of a permutation $\sigma \in S_n$ in the symmetric group on $n$ letters is the list $m_1(\sigma),m_2(\sigma),\ldots,m_n(\sigma)$ where $m_i(\sigma)$ is the number of $i$-cycles in $\sigma$'s cycle decomposition.
\begin{enumerate}
\item Prove that two permutations in $S_n$ are in the same conjugacy class if and only if they have the same cycle type.
\item Prove that the cardinality of the conjugacy class of $\sigma \in S_n$ is $\displaystyle \frac{n!}{1^{m_1}\, m_1! \, 2^{m_2}\, m_2! \, \cdots \, n^{m_n}\, m_n!}$ where $m_i = m_i(\sigma)$ are the numbers in the cycle type of $\sigma$.
\end{enumerate}

\item Let $G$ be a finite group of order $pq$ for distinct primes $p < q$. Prove that $G$ is not simple, i.e., that it has a normal subgroup $N\trianglelefteq G$ other than $\{e\}$ and $G$. \\ {\bf Hint}: use the Sylow theorems; specifically, show that any Sylow $q$-subgroup is normal in $G$.

\end{enumerate}


\end{document}
