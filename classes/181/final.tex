\documentclass[11pt]{article}
\usepackage[top=1in, bottom=1in, left=1in, right=1in]{geometry}

\usepackage{amsmath}
\usepackage{amssymb}
\usepackage{graphicx}

\title{Final Exam, 5/3 \\Math 181 (Discrete Structures), Spring 2024}
\date{}

\begin{document}

\maketitle

\thispagestyle{empty}

\vspace{-1.8cm}

Each problem is worth 10 points, for a total of 80 points. You have 90 minutes to do the exam. Remember to \emph{show your work} and \emph{explain your answers} on all problems!

\begin{enumerate}

\item Let $A=\{1,3,5,7\}$, $B= \{2,4,5,7\}$ and $C = \{2,3,5,6\}$.
\begin{enumerate}
\item Draw a Venn diagram for this situation.
\item Let $X = (A \setminus B) \cup (B \setminus A)$. Shade the area of the Venn diagram corresponding to $X$.
\item Write the elements of $C \cap X$.
\end{enumerate}

\item Convert the following argument to symbolic form and decide (with explanation) if it's valid. 
\smallskip

Hypotheses: If I'm hungry or I'm thirsty then I go to the cafeteria. I'm thirsty. \\ Conclusion: I go to the cafeteria or I go to my office.

\item Give a proof of this theorem: ``For any sets $X$, $Y$, and $Z$, if $X \subseteq Y$ then $X \cap Z \subseteq Y \cap Z$.''

\item Prove by induction that, for all $n \geq 1$,
\[ 1\times 1! + 2 \times 2! + 3 \times 3! + \cdots + n \times n! = (n+1)! - 1.\]

\item Let $X = \{0,1,2\}$. Define functions $f\colon X \to X$ and $g\colon X \to X$ by letting
\[ f(x) = 2x+1 \mod 3 \qquad \textrm{ and } \qquad g(x) = x^2 \mod 3\]
for all $x \in X$.
\begin{enumerate}
\item Draw the arrow diagrams for $f$, for $g$, and for $f\circ g$.
\item Which of $f$, $g$, and $f\circ g$ are bijections? Explain.
\end{enumerate}

\item For integers $a$ and $b$, we say that $a$ \emph{divides} $b$ if there is some integer $c$ such that $b = c \times a$. Define a relation $R$ on the set $\{1,2,3,\ldots\}$ of positive integers where we have $a \; R \; b$ if and only if $a$ divides $b$. For each of the following four properties, explain whether the relation $R$ has that property or not: (i) reflexive, (ii) symmetric, (iii) anti-symmetric, and (iv) transitive.

\item \begin{enumerate}
\item How many rearrangements of the word ALASKA start with an A?
\item How many rearrangements of ALASKA end with an S?
\item How many rearrangements of ALASKA start with an A or end with an S (or both)?
\end{enumerate}

\item Recall that Pascal's triangle of binomial coefficients $C(n,k)$ begins:
\begin{center}
\begin{tabular}{c c c c c c c}
 & & & 1 & & & \\
 & & 1 & & 1 & & \\
 & 1 && 2 && 1 & \\
 1 && 3 && 3 && 1
\end{tabular}
\end{center}
\begin{enumerate}
\item Write down the next three rows of Pascal's triangle, i.e., the rows for $n=4$, $5$, and $6$.
\item Using part (a): expand the polynomial $(x+y)^6$.
\item Using part (a): how many three element subsets of a five element set are there?
\end{enumerate}

\end{enumerate}

\end{document}