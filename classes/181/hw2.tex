\documentclass[11pt]{article}
\usepackage[top=1in, bottom=1in, left=1in, right=1in]{geometry}

\usepackage{amsmath}
\usepackage{amssymb}
\usepackage{graphicx}

\title{Homework \#2, Due: 1/24 \\Math 181 (Discrete Structures), Spring 2024}
\date{}

\begin{document}

\maketitle

\thispagestyle{empty}

\vspace{-1cm}

Problem 1 is worth 4 points, Problem 2 is worth 2 points, and Problem 3 is worth 4 points, for a total of 10 points. Remember to \emph{show your work} and \emph{explain your answers} on all problems!

\begin{enumerate}
\item Write the truth tables of the following compound propositions:
\begin{enumerate}
\item $q \wedge \neg p$
\item $(p \wedge q) \vee \neg q$
\end{enumerate}

\item Let $p$, $q$, and $r$ be the following propositions:
\begin{align*}
p &: \textrm{You took a math class this semester.} \\
q &: \textrm{You took a computer science class this semester.} \\
r &: \textrm{You took a physics class this semester.}
\end{align*}
Represent the following propositions symbolically in terms of $p$, $q$, and $r$:
\begin{enumerate}
\item ``You took a math class and a physics class this semester.''
\item ``You took a math or computer science class this semester, and you did not take a physics class this semester.''
\end{enumerate}

\item \begin{enumerate}
\item Write the converse of ``If Maria is looking at the Eiffel Tower, then she is in France.''
\item Write the contrapositive of ``If Maria is looking at the Eiffel Tower, then she is in France.''
\item Is the converse of $p \to q$ logically equivalent to $p \to q$? Explain (for instance, by giving an example, or writing a truth table).
\item Is the contrapositive of $p \to q$ logically equivalent to $p \to q$? Explain (for instance, by giving an example, or writing a truth table).
\end{enumerate}
\end{enumerate}

\end{document}
