\documentclass[11pt]{article}
\usepackage[top=1in, bottom=1in, left=1in, right=1in]{geometry}

\usepackage{amsmath}
\usepackage{amssymb}
\usepackage{graphicx}

\title{Homework \#9, Due: 3/27 \\Math 181 (Discrete Structures), Spring 2024}
\date{}

\begin{document}

\maketitle

\thispagestyle{empty}

\vspace{-1cm}

Problem 1 is worth 4 points, Problem 2 is worth 2 points, and Problem 3 is worth 4 points, for a total of 10 points. Remember to \emph{show your work} and \emph{explain your answers} on all problems!

\begin{enumerate}

\item Let $X=\{a,b\}$ and recall that $X^*$ denotes the set of all strings over the alphabet $X$. \\ Define a function $f\colon X^* \to X^*$ by letting $f(\alpha)$ be the result of simultaneously replacing each $a$ with a $b$, and each $b$ with an $a$, in the string $\alpha \in X^*$. For instance $f(aab) = bba$.
\begin{enumerate}
\item Write what $f(a)$, $f(bb)$, $f(baba)$, and $f(\lambda)$ are. (Recall $\lambda \in X^*$ denotes the null string.)
\item Recall that for strings $\alpha, \beta \in X^*$, we use $\alpha \beta$ to mean the concatenation of $\alpha$ and $\beta$. Express $f(\alpha\beta)$ in terms of $f(\alpha)$ and $f(\beta)$.
\item What is $f( f(\alpha))$ for a string $\alpha\in X^*$?
\item Is $f$ one-to-one? Is $f$ onto? Explain.
\end{enumerate}

\item Let $R$ be the relation on $\{1,2,3,4\}$ given by $R = \{(1,2),(2,3),(3,4),(4,1)\}$. Draw the digraph representation of $R$. Also draw the digraph representation of $R^{-1}$ (the inverse relation).

\item Let $R$ be the relation on the integers $\mathbb{Z}$ where $(x,y) \in R$ if and only if $x-y$ is even.
\begin{enumerate}
\item Is $R$ reflexive? Explain.
\item Is $R$ symmetric? Explain.
\item Is $R$ anti-symmetric? Explain.
\item Is $R$ transitive? Explain.
\end{enumerate}

\end{enumerate}

\end{document}
