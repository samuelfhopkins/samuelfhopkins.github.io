\documentclass[11pt]{article}
\usepackage[top=1in, bottom=1in, left=1in, right=1in]{geometry}

\usepackage{amsmath}
\usepackage{amssymb}
\usepackage{graphicx}

\title{Midterm \#1 Study Guide \\Math 181 (Discrete Structures), Spring 2023}
\date{}

\begin{document}

\maketitle

\thispagestyle{empty}

\vspace{-1cm}

\begin{enumerate}
\item Sets [\S1.1] 
\begin{enumerate}
\item sets of numbers (integers $\mathbb{Z}$ and real numbers $\mathbb{R}$), set-builder notation, subsets ($A \subseteq B$)
\item operations of union ($A \cup B$), intersection ($A \cap B$), difference ($A \setminus B$), complement ($A^c$)
\item representing sets via Venn diagrams
\item ordered pairs $(x,y)$ and the (Cartesian) product $X \times Y$ of two sets $X$ and $Y$
\end{enumerate}

\item Logical propositions [\S1.2, 1.3]
\begin{enumerate}
\item operations of ``or'' ($p \vee q$), ``and'' ($p \wedge q$), ``not'' ($\neg p$)
\item truth tables for compound propositions 
\item conditional a.k.a.~implication a.k.a.~``if... then...'' ($p \to q$)
\item biconditionals ($p \leftrightarrow q$) and logical equivalence ($\equiv$)
\item converse $q \to p$ and contrapositive $\neg q \to \neg p$ of an implication $p\to q$ \\
(contrapositive is logically equivalent to original implication; converse is not!)
\end{enumerate}

\item Logical arguments [\S1.4]
\begin{enumerate}
\item converting an argument from words to symbolic form and vice-versa
\item proving validity using truth tables
\item proving validity using the rules of inference and logical equivalences
\item common forms of invalid arguments a.k.a.~fallacies
\end{enumerate}

\item Quantifiers [\S1.5, 1.6]
\begin{enumerate}
\item propositional formulas ($P(x)$) and domains of discourse ($D$)
\item universal ($\forall x \; P(x)$) and existential ($\exists x \; P(x)$) quantifiers
\item DeMorgan's Laws: $\neg (\forall x \; P(x)) \equiv \exists x \; \neg P(x)$ and $\neg( \exists x \; P(x)) \equiv \forall x \; \neg P(x)$
\item nested quantifiers and order of quantifiers ($\forall x \exists y \; P(x,y) \not \equiv \exists y \forall x \; P(x,y)$)
\end{enumerate}

\item Proofs [\S2.1]
\begin{enumerate}
\item two basic mathematical systems: the theory of integers; the theory of sets
\item direct proofs for theorems of form ``$\forall x_1,\ldots,x_n \; \textrm{if $P(x_1,\ldots,x_n)$ then $Q(x_1,\ldots,x_n)$}$''
\item counterexamples to universally quantified statements
\end{enumerate}

\end{enumerate}

\end{document}