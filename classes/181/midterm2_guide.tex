\documentclass[11pt]{article}
\usepackage[top=1in, bottom=1in, left=1in, right=1in]{geometry}

\usepackage{amsmath}
\usepackage{amssymb}
\usepackage{graphicx}

\title{Midterm \#2 Study Guide \\Math 181 (Discrete Structures), Spring 2024}
\date{}

\begin{document}

\maketitle

\thispagestyle{empty}

\vspace{-1cm}

\begin{enumerate}
\item Indirect proofs [\S2.2] 
\begin{enumerate}
\item proof by contrapositive: to prove $p \to q$, prove $\neg q \to \neg p$ instead
\item proof by contradiction: assume negation of statement, and deduce contradiction ($r \wedge \neg r$)
\end{enumerate}

\item Mathematical induction [\S2.4, 2.5]
\begin{enumerate}
\item basic structure of inductive proofs: base case $P(1)$, and induction step $P(n) \to P(n+1)$
\item proving $\forall (n \in \mathbb{Z}_{>0}) \; P(n)$ by induction, especially when $P(n)$ is an algebraic formula
\item finding patterns to guess formulas involving $n$ which can then be proved by induction
\item the strong form of mathematical induction: can use $P(k)$ for all $k < n$ to prove $P(n)$
\end{enumerate}

\item Functions [\S3.1]
\begin{enumerate}
\item ways to view a function $f\colon X \to Y$: rule to convert input $x\in X$ to output $y=f(x)\in Y$; set of ordered pairs $(x,y)$; arrow diagram from $X$ to $Y$
\item one-to-one, onto, and bijective functions
\item composition of functions, and inverse functions
\item modular arithmetic functions like $f(x) = x \mod n$
\end{enumerate}

\item Sequences and strings [\S3.2]
\begin{enumerate}
\item finite and infinite sequences: ordered list of elements of some set
\item set of strings $X^*$ on a finite alphabet $X$, the null string $\lambda \in X^*$, concatenation of strings
\item subsequences (not necessarily consecutive) versus substrings (consecutive)
\end{enumerate}

\item Relations [\S3.4, 3.5]
\begin{enumerate}
\item digraph representation of a relation $R$ on a set $X$
\item properties that $R$ can have: reflexive, symmetric, anti-symmetric, transitive
\item partial order (reflexive, anti-symmetric, transitive): way to ``compare'' things in $X$
\item equivalence relation (reflexive, symmetric, transitive): way to say certain things in $X$ are ``the same''; corresponds to a partition of $X$ into equivalence classes
\end{enumerate}

\item Basic counting principles [\S6.1]
\begin{enumerate}
\item multiplication principle: total $\#$ of possibilities = product of $\#$ of choices at each step
\item addition principle: size of union of \emph{disjoint} sets is sum of sizes of the sets
\item principle of inclusion and exclusion: $\#(X \cup Y) = \#X + \#Y - \# (X \cap Y)$
\end{enumerate}

\end{enumerate}

\end{document}