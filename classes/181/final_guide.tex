\documentclass[11pt]{article}
\usepackage[top=1in, bottom=1in, left=1in, right=1in]{geometry}

\usepackage{amsmath}
\usepackage{amssymb}
\usepackage{graphicx}

\title{Final Exam Study Guide \\Math 181 (Discrete Structures), Fall 2022}
\date{}

\begin{document}

\maketitle

\vspace{-1.5cm}

\pagestyle{empty}
\thispagestyle{empty}

\begin{enumerate}
\item Sets [\S1.1] 
\begin{enumerate}
\item sets of numbers (integers $\mathbb{Z}$ and real numbers $\mathbb{R}$), set-builder notation, subsets ($A \subseteq B$)
\item operations of union ($A \cup B$), intersection ($A \cap B$), difference ($A \setminus B$), complement ($A^c$)
\item representing sets via Venn diagrams
\end{enumerate}

\item Logical propositions [\S1.2, 1.3]
\begin{enumerate}
\item operations of ``or'' ($p \vee q$), ``and'' ($p \wedge q$), ``not'' ($\neg p$)
\item truth tables for compound propositions 
\item conditional a.k.a.~implication a.k.a.~``if... then...'' ($p \to q$)
\item biconditionals ($p \leftrightarrow q$) and logical equivalence ($\equiv$)
\item converse $q \to p$ and contrapositive $\neg q \to \neg p$ of an implication $p\to q$ \\
(contrapositive is logically equivalent to original implication; converse is not!)
\end{enumerate}

\item Logical arguments [\S1.4]
\begin{enumerate}
\item converting an argument from words to symbolic form and vice-versa
\item proving validity using truth tables
\item proving validity using the rules of inference and logical equivalences
\item common forms of invalid arguments a.k.a.~fallacies
\end{enumerate}

\item Quantifiers [\S1.5, 1.6]
\begin{enumerate}
\item propositional formulas ($P(x)$) and domains of discourse ($D$)
\item universal ($\forall x \; P(x)$) and existential ($\exists x \; P(x)$) quantifiers
\item DeMorgan's Laws: $\neg (\forall x \; P(x)) \equiv \exists x \; \neg P(x)$ and $\neg( \exists x \; P(x)) \equiv \forall x \; \neg P(x)$
\item nested quantifiers and order of quantifiers ($\forall x \exists y \; P(x,y) \not \equiv \exists y \forall x \; P(x,y)$)
\end{enumerate}

\item Proofs [\S2.1]
\begin{enumerate}
\item two basic mathematical systems: the theory of integers; the theory of sets
\item direct proofs for theorems of form ``$\forall x_1,\ldots,x_n \; \textrm{if $P(x_1,\ldots,x_n)$ then $Q(x_1,\ldots,x_n)$}$''
\item counterexamples to universally quantified statements
\end{enumerate}

\item Indirect proofs [\S2.2] 
\begin{enumerate}
\item proof of by contrapositive: to prove $p \to q$, prove $\neg q \to \neg p$ instead
\item proof by contradiction: assume negation of statement, and deduce contradiction ($r \wedge \neg r$)
\end{enumerate}

\item Mathematical induction [\S2.4, 2.5]
\begin{enumerate}
\item basic structure of inductive proofs: base case $P(1)$, and induction step $P(n) \to P(n+1)$
\item proving $\forall (n \in \mathbb{Z}_{>0}) \; P(n)$ by induction, especially when $P(n)$ is an algebraic formula
\item finding patterns to guess formulas involving $n$ which can then be proved by induction
\item the strong form of mathematical induction: can use $P(k)$ for all $k < n$ to prove $P(n)$
\end{enumerate}

\item Functions [\S3.1]
\begin{enumerate}
\item ways to view a function $f\colon X \to Y$: rule to convert input $x\in X$ to output $y=f(x)\in Y$; set of ordered pairs $(x,y)$; arrow diagram from $X$ to $Y$
\item one-to-one, onto, and bijective functions
\item composition of functions, and inverse functions
\item modular arithmetic functions $f(x) = x \mod n$
\end{enumerate}

\item Sequences and strings [\S3.2]
\begin{enumerate}
\item finite and infinite sequences: ordered list of elements of some set
\item set of strings $X^*$ on some finite alphabet $X$, and the null string $\lambda \in X^*$
\item subsequences (not necessarily consecutive) versus substrings (consecutive)
\end{enumerate}

\item Relations [\S3.4, 3.5]
\begin{enumerate}
\item digraph representation of a relation $R$ on a set $X$
\item properties that $R$ can have: reflexive, symmetric, anti-symmetric, transitive
\item partial order (reflexive, anti-symmetric, transitive): way to ``compare'' things in $X$
\item equivalence relation (reflexive, symmetric, transitive): way to say certain things in $X$ are ``the same''; corresponds to a partition of $X$ into equivalence classes
\end{enumerate}

\item Basic counting principles [\S6.1]
\begin{enumerate}
\item multiplication principle: total $\#$ of possibilities = product of $\#$ of choices at each step
\item addition principle: size of union of \emph{disjoint} sets is sum of sizes of the sets
\item principle of inclusion and exclusion: $\#(X \cup Y) = \#X + \#Y - \# (X \cap Y)$
\end{enumerate}

\item Permutations and combinations [\S6.2, 6.3]
\begin{enumerate}
\item number of permutations (=orderings) of $n$ element set is $n! = n \times (n-1) \times \cdots \times 1$, and number of $k$-permutations (=orderings of $k$ element subsets) is $P(n,k) = n!/(n-k)!$
\item for rearrangements of word with repeated letters like MISSISSIPPI use $n! / (k_1! k_2! \cdots k_m !)$
\item number of $k$-combinations (= $k$ element subsets) of $n$ element set is the binomial coefficient, a.k.a.~``$n$ choose $k$'' number, $C(n,k) = n!/( k! \cdot (n-k)! )$
\item for selections of $k$ things from $n$ things with repeats allowed use $C(k+n-1,k)$
\end{enumerate}

\item Binomial coefficients [\S 6.7]
\begin{enumerate}
\item Binomial Theorem: $(x+y)^n = \sum_{k=0}^{n} C(n,k) x^k y^{n-k}$
\item Pascal's Triangle of $C(n,k)$, defined by recurrence $C(n+1,k) = C(n,k) + C(n,k-1)$
\end{enumerate}


\item Pigeonhole principle [\S6.8]
\begin{enumerate}
\item if you place $n$ pigeons in $n-1$ holes, at least one hole has at least two pigeons
\end{enumerate}

\end{enumerate}

\end{document}