\documentclass[11pt]{article}
\usepackage[top=1in, bottom=1in, left=1in, right=1in]{geometry}

\usepackage{amsmath}
\usepackage{amssymb}
\usepackage{graphicx}

\title{Homework \#11, Due: 4/19 \\Math 181 (Discrete Structures), Spring 2023}
\date{}

\begin{document}

\maketitle

\thispagestyle{empty}

\vspace{-1cm}

Problem 1 is worth 4 points (2 pts each part), and Problem 2 is worth 6 points (2 pts each part), for a total of 10 points. Remember to \emph{show your work} and \emph{explain your answers} on all problems!

\begin{enumerate}

\item In a standard deck of playing cards, cards have two qualities:
\begin{itemize}
\item a \emph{rank}: 2, 3, 4, 5, 6, 7, 8, 9, 10, Jack, Queen, King, or Ace;
\item a \emph{suit}: Spades ($\spadesuit$), Hearts ($\heartsuit$), Diamonds ($\diamondsuit$), or Clubs ($\clubsuit$).
\end{itemize}
There are $13$ ranks and $4$ suits, and each combination of rank and suit appears exactly once. So there are a total of $13 \times 4 = 52$ cards. A \emph{poker hand} consists of any $5$ of these $52$ cards. We saw in class that there are $C(52,5) = 52!/(5! \cdot 47!)=2,598,960$ different poker hands.
\begin{enumerate}
\item A poker hand is called \emph{four of a kind} if it consists of all four cards of one rank, plus any other card. For instance: $8\spadesuit \; 8\heartsuit \; 8\diamondsuit \; 8\clubsuit \; 3\diamondsuit$. How many four of a kind hands are there?
\item A poker hand is called a \emph{full house} if it consists of three of the cards of one rank, and two of the cards of another rank. For instance: $5\spadesuit \; 5\heartsuit \; 5\clubsuit \; \mathrm{J}\heartsuit \; \mathrm{J}\diamondsuit$. How many full house hands are there?
\end{enumerate}

\item \begin{enumerate}
\item How many rearrangements of the word LOLLYPOP are there?
\item How many rearrangements of LOLLYPOP start with a Y or end with a~P (or both)? \\ {\bf Hint}: remember the Principle of Inclusion-Exclusion, $\#(X\cup Y) = \#X + \#Y - \#(X\cap Y)$.
\item How many rearrangements of LOLLYPOP have the two O's adjacent? \\ {\bf Hint}: to make the O's adjacent, you can treat them as a single character ``OO.''
\end{enumerate}

\end{enumerate}

\end{document}
