\documentclass[11pt]{article}
\usepackage[top=1in, bottom=1in, left=1in, right=1in]{geometry}

\usepackage{amsmath}
\usepackage{amssymb}
\usepackage{graphicx}

\title{Midterm \#2, 11/16 \\Math 181 (Discrete Structures), Fall 2022}
\date{}

\begin{document}

\maketitle

\thispagestyle{empty}

\vspace{-1cm}

Each problem is worth 10 points, for a total of 50 points. You have 50 minutes to do the exam. Remember to \emph{show your work} and \emph{explain your answers} on all problems!

\begin{enumerate}

\item Prove the following theorem: ``If the product of two integers is even, then at least one of these two integers must be even.'' Use proof by contrapositive or proof by contradiction.

\item Prove by induction that $1+3+5 + \cdots + (2n-1) = n^2$ for any integer $n\geq 1$. (The left-hand side of the identity is the sum of all odd positive integers less than or equal to $2n-1$.)

\item Let $X = \{0,1,2,3\}$. Let the function $f\colon X \to X$ be given by $f(x) = 3x \mod 4$ for all $x \in X$. Draw the arrow diagram of $f$. Is $f$ one-to-one? Is $f$ onto?

\item Let $X = \{a,b,c\}$ and define a relation $R$ on the set $X^*$ of strings over $X$ where for $\alpha, \beta \in X^*$ we have $\alpha \; R \; \beta$ if and only if $\alpha$ and $\beta$ have the same first letter. For example, $abc \; R \; acabb$ and $bb \; R \; bca$. For the null string $\lambda \in X^*$ (which has no first letter), we declare that $\lambda$ is the only string that relates or is related to~$\lambda$ according to $R$. Explain why this relation $R$ on $X^*$ is an equivalence relation, and describe all the equivalence classes of $R$.

\item Let $A = \{1,2\}$ and $C = \{1,2,3,4,5,6\}$. How many sets $B$ with $A \subseteq B \subseteq C$ are there? Explain your answer, for instance by referencing a counting principle.


\end{enumerate}

\end{document}