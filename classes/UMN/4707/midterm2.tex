\documentclass[11pt]{article}

\usepackage{amsmath}
\usepackage{amssymb}

\usepackage{graphicx}
\usepackage{tikz}

\usepackage[all]{xy}

\begin{document}

\begin{center}
{\bf Math 4707: Intro to combinatorics and graph theory \\
Spring 2021, Instructor: Sam Hopkins \\
Midterm exam 2- Due Wednesday, March 31st}
\end{center}

{\bf Instructions}: There are 5 problems, worth 20 points each, totaling 100 points. This is an open book, open library, open notes, open web, take-home exam, but you are not allowed to to interact with anyone (including online forums) except for me, the instructor. As always, in order to earn points you need to carefully \emph{explain your answer}.

\begin{enumerate}

\item (20 points total) In this problem a \emph{forest} is a graph containing no cycles, a \emph{tree} is a connected forest, and a \emph{leaf} is a vertex of degree~$1$. 
\begin{enumerate} 
\item (10 points) Prove that a tree having at least one vertex of degree~$d$ always has at least $d$ leaves. 
\item (5 points) Prove that a tree $T$ with $n$ vertices has \[ \sum_{v} (\mathrm{deg}_T(v)-1)=n-2,\] where the sum is over all the vertices $v$ of $T$. 
\item (5 points) Given a forest with $n$ vertices and $c$ connected components, how many edges will it contain (as a function of $n$ and~$c$)? \end{enumerate}

\item (20 points total; 5 points each part) For each $n=1,2,\ldots$, define a bipartite graph $G_n$ on vertex set $V=X\sqcup Y$ where $X = \{x_1,x_2,\ldots,x_n\}$ and $Y = \{y_1,y_2,\ldots,y_n\}$, with edge set
\[ E := \{\{x_i,y_j\}\colon i = 1,2,\ldots, n \textrm{ and } j=1,2,\ldots,n, \textrm{ and } i\neq j \}.\]
Thus $G_n$ has $2n$ vertices and $n(n-1)$ edges.

Explain, with proof, exactly for which values of $n=1,2,3,\ldots$ does the graph $G_n$ contain $\ldots$ 
\begin{enumerate} \item $\ldots$ a spanning tree? 
\item $\ldots$ an Eulerian circuit? 
\item $\ldots$ a perfect matching? 
\item $\ldots$ a Hamiltonian cycle? \end{enumerate}

\pagebreak

\item (20 points) The student newspaper of your university has a certain number of sports reporters: Sam, Emily, Lauren, $\ldots$. The university plays a certain number of sports: Basketball, Soccer, Hockey, $\ldots$. Each sports reporter is asked to list which sports they would be interested in covering. You are not told the number of reporters nor the number of sports. But what you do know is that: \begin{itemize} \item each reporter lists exactly 3 sports they are interested in covering; \item and each sport has exactly 3 reporters who listed that sport. \end{itemize} Explain how you know that there is a way of assigning reporters to sports they listed, so that each reporter is assigned exactly one sport and each sport has exactly one reporter assigned to it. 

[For this problem: if you use a result that is stated but not proved in the textbook, you should prove that result yourself; but feel free to use results we proved in lecture.]

\item (20 points) Find (e.g., darken) a minimum cost spanning tree $T$ in the following graph, whose edges have been labeled by their costs. What is its cost? Explain in one or two lines how you know that it achieves the minimum cost.
\[ \xymatrix{
\bullet \ar@{-}[r]^5 \ar@{-}[d]^3 & \bullet  \ar@{-}[r]^2  \ar@{-}[d]^2 & \bullet  \ar@{-}[r]^2  \ar@{-}[d]^3 & \bullet  \ar@{-}[r]^3  \ar@{-}[d]^3 & \bullet  \ar@{-}[r]^3  \ar@{-}[d]^3 & \bullet  \ar@{-}[r]^2  \ar@{-}[d]^1 &  \ar@{-}[d]^1 \bullet \\
\bullet \ar@{-}[r]^1 \ar@{-}[d]^2 & \bullet  \ar@{-}[r]^5  \ar@{-}[d]^4 & \bullet  \ar@{-}[r]^2  \ar@{-}[d]^2 & \bullet  \ar@{-}[r]^7  \ar@{-}[d]^1 & \bullet  \ar@{-}[r]^3  \ar@{-}[d]^2 & \bullet  \ar@{-}[r]^3  \ar@{-}[d]^1 &  \ar@{-}[d]^3 \bullet \\
\bullet \ar@{-}[r]^1 \ar@{-}[d]^2 & \bullet  \ar@{-}[r]^3  \ar@{-}[d]^3 & \bullet  \ar@{-}[r]^3  \ar@{-}[d]^2 & \bullet  \ar@{-}[r]^3  \ar@{-}[d]^6 & \bullet  \ar@{-}[r]^3  \ar@{-}[d]^3 & \bullet  \ar@{-}[r]^4  \ar@{-}[d]^5 &  \ar@{-}[d]^3 \bullet \\
\bullet \ar@{-}[r]^2 & \bullet  \ar@{-}[r]^4 & \bullet  \ar@{-}[r]^3 & \bullet  \ar@{-}[r]^8 & \bullet  \ar@{-}[r]^7 & \bullet  \ar@{-}[r]^1 & \bullet
} \]

\item (20 points) Exercise 7.2.11 on p.~134 of our text: Prove that a \emph{simple} graph $G$ (i.e., one with \emph{no multiple edges and no self-loops}) having $n$ vertices and strictly more than $\binom{n-1}{2}$ edges must be connected.

\end{enumerate}


\end{document}
