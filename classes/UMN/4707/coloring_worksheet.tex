\documentclass[11pt]{article}

\usepackage{amsmath}
\usepackage{amssymb}

\usepackage{graphicx}

\usepackage{ytableau}

\title{Graph coloring, \\Math 4707, Spring 2021}
\date{}

\begin{document}

\maketitle
\thispagestyle{empty}

Recall that for a graph $G$, the \emph{chromatic number} of $G$, denoted $\chi(G)$, is the smallest number of colors needed to (properly) color the vertices of $G$. Graphs with $\chi(G)=2$ are called \emph{bipartite}.

If $G$ has a subgraph isomorphic to $K_m$, the complete graph on $m$ vertices, then $\chi(G) \geq m$ because it requires $m$ colors to color even that subgraph.

\begin{enumerate}
\item Give an example of a graph which does not contain a subgraph isomorphic to $K_3$ but with $\chi(G) \geq 3$.
\item Give an example of a graph which does not contain a subgraph isomorphic to $K_4$ but with $\chi(G) \geq 4$.
\item {\bf Challenge}: for each $m \geq 3$, give an example of a graph $G$ which does not contain a subgraph isomorphic to $K_m$ but with $\chi(G) \geq m$.
\end{enumerate}

{\bf Remark}: In fact, much more is true. The \emph{girth} of a graph $G$ is the size of the smallest cycle in $G$. A classic result of Erd\H{o}s (that is beyond what we'll prove in this class) says that for any $g,m$, there exists a graph $G$ with girth $\geq g$ and $\chi(G) \geq m$.
\medskip

Let $\Delta(G)$ denote the maximum degree of $G$. We saw a simple proof by induction that $\chi(G) \leq \Delta(G)+1$.

\begin{enumerate}
\setcounter{enumi}{3}
\item Show that the bound just mentioned is sharp: for each $d \geq 2$, give an example of a graph with $\Delta(G)=d$ and $\chi(G) = d+1$. How many examples can you think of?
\end{enumerate}

{\bf Remark}: \emph{Brooks' theorem} says that the only $G$ with $\chi(G) = \Delta(G)+1$ are the ``obvious'' examples.

\begin{enumerate}
\setcounter{enumi}{4}
\item Let $G$ be a bipartite graph on $n$ vertices. How big can $\Delta(G)$ be?
\item For each $d \geq 1$, give an example of a bipartite graph $G$ for which the \emph{minimum degree} of $G$ is $d$.
\end{enumerate}

\end{document}
