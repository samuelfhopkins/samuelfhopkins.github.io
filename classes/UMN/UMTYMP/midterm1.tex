\documentclass[11pt]{article}

\usepackage{amsmath}
\usepackage{amssymb}

\usepackage{graphicx}
\usepackage{tikz}

\usepackage{ytableau}

\begin{document}

\thispagestyle{empty}

\begin{center}
{\bf Math 4990: Intro to combinatorics and graph theory \\
Fall 2020, Sam Hopkins \\
Midterm exam 1- Due Tuesday Oct. 13th}
\end{center}

{\bf Instructions}: There are 5 problems, worth 20 points each, totaling 100 points. This is an open book, open library, open notes, open web, take-home exam, but you are not allowed to to interact with anyone (including online forums) except for me, the instructor. As always, in order to earn points you need to carefully \emph{explain your answer}.

\medskip

Throughout, $[n] :=\{1,2,\ldots,n\}$.

\begin{enumerate}

\item (20 points total) \begin{enumerate}
\item (10 points) How many rearrangements (i.e.,~anagrams) are there of the letters in the word ``COMMITTEE''?
\item (10 points) What is the probability that the first two letters are the same in a (uniformly) random such rearrangement?
\end{enumerate}

\item (20 points) Let $A(n,k)$ denote the number of set partitions of $[n]$ into $k$ parts for which every part has size at least $2$. Prove the recurrence $A(n,k) = k\cdot A(n-1,k) + (n-1)\cdot A(n-2,k-1)$.

\item (20 points) Find all triples $(a,b,c)$ of positive integers $a \geq b \geq c \geq 1$ such that
\[\binom{a}{b}\binom{b}{c} = 2\binom{a}{c}.\]

\item (20 points total) Let $\pi \in S_n$ be a permutation with $k$ cycles of sizes $c_1,c_2,\ldots, c_k$. Let $m$ be the smallest positive integer with $\pi^m(i)=i$ for all $i \in [n]$. \begin{enumerate}
\item (10 points) Explain what $m$ is in terms of $c_1,\ldots,c_k$.
\item (10 points) Give an upper bound for $m$ in terms of $n$ only (and not $k$, $c_1,\ldots,c_k$).
\end{enumerate} 

\item (20 points) Let $\mathcal{F}$ be a collection of subsets of $[n]$. Prove that if $\mathcal{F}$ contains at least $2^{n-1}+1$ subsets, then there are two subsets $T,S \in \mathcal{F}$ for which $T\subset S$.

\end{enumerate}


\end{document}
