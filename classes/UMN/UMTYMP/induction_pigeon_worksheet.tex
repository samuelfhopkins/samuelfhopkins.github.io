\documentclass[11pt]{article}
\usepackage[tmargin=0.5in,bmargin=1in,rmargin=1.5in,lmargin=1.5in]{geometry}


\usepackage{amsmath}
\usepackage{amssymb}

\usepackage{graphicx}
\usepackage{tikz}

\usepackage{ytableau}

\title{Induction and the pigeonhole principle, \\UMTYMP Advanced Topics, Fall 2020}
\date{}


\begin{document}


\maketitle

\thispagestyle{empty}

Taken from Chapters 1 and 2 of B\'{o}na's ``Walk Through Combinatorics'' (so you can look up the solutions there).

\medskip

\begin{enumerate}

\item Prove that for all natural numbers $n$, the number $a(n) = n^3 + 11n$ is divisible by $6$.
\item Let $a_0 = 1$, and let $a_{n+1} = 2 \sum_{i=0}^{n} a_i$ for all non-negative integers $n$. Find an explicit formula for $a_n$.
\item We cut a square into four smaller squares, then we cut some of the obtained small squares into four smaller squares, and so on. Prove that at any given point of time during this operation, the number of all squares we have is of the form $3m+1$.

\medskip

\medskip

\item Find all triples of positive integers $a < b < c$ for which \[\frac{1}{a}+\frac{1}{b}+\frac{1}{c}=1.\]

\item There are four heaps of stones in our backyard. We rearrange them into five heaps. Prove that at least two stones are placed into a smaller heap.

\end{enumerate}

\end{document}
